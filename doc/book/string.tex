\chapter{String Processing}

In addition to its groundbreaking expression evaluation, by combining
compelling string processing features from its ancestors Icon and
\index{SNOBOL4}SNOBOL4, Unicon provides some of the most flexible and
readable built-in string processing facilities found in any language.
If you are used to string processing in a mainstream language, hold on
to your hat: things are about to get interesting.

In this chapter you will learn

\begin{itemize}
\item How to manipulate strings and sets of characters
\item the string scanning \index{control structure}control structure,
used to match patterns directly in code
\item the pattern type, used to match patterns constructed as data
\item How to write custom \index{pattern matching}pattern matching
primitives, with \index{backtracking}backtracking
\item techniques for matching regular expressions and context free grammars
\end{itemize}

\section{The String and Cset Types}

All mainstream programming languages have a string type, but the
details of Unicon's string type set it apart from other languages.
And almost no other mainstream languages feature a data type
dedicated to character sets, which are quite useful.

\subsection{String Indexes}

You have already seen string literals delimited by double quotes, and
the most common operators that work on strings: the size of a string is
given by the unary \texttt{*} operator, substrings can be picked out
with square-bracketed indexes, and two strings can be concatenated with the
\texttt{{\textbar}{\textbar}} operator. It is time for a
deeper explanation of the meaning of indexes as they are used with
strings and lists.

\index{string!indexes 1 based}Indexes in a string refer to the positions
{\em between\/} characters. The positions are numbered starting from 1. The
index 0 refers to the position after the last character in the string,
and negative indices count from the right side of the string:


\begin{center}
\includegraphics[width=3.6075in,height=1.0417in]{ub-img/ub-img7.png}
\end{center}
\vspace{-0.25cm}{\sffamily\bfseries Figure 3-1:}
{\sffamily Positive and Negative String Indices}

\bigskip

The expression \index{slice!string s[i:j]}\texttt{s[i:j]} refers to the
\index{substring}substring of \texttt{s} that lies between positions
\texttt{i} and \texttt{j}. If either \texttt{i} or j is not a valid
index into \texttt{s}, the expression fails. The expression
\texttt{s[k]} is short for \texttt{s[k:k+1]} and refers to a single
\index{character}character at position \texttt{k}. The expression
\texttt{s[k+:n]} is the substring of length \texttt{n} starting at
position \texttt{k}. If \texttt{s} is the string
\texttt{"hello, world!"} then the
expressions

\iconcode{
s[7] := " puny " \\
s[13:18] := "earthlings"
}

\noindent
change \texttt{s} into \texttt{"hello, puny
earthlings!"}, illustrating the ease with which
insertions and substitutions are made. The first assignment changes the
string to \texttt{"hello, puny world!"},
replacing a single character with six characters and thereby increasing
its length. The second assignment operates on the modified string,
replacing \texttt{"world"} with
\texttt{"earthlings"}.

Strings are values, just like numbers; if you copy a string and then
work on the copy, the original will be left unchanged:

\iconcode{
s := "string1" \\
new\_s := s \\
new\_s[7] := "2"
}

Now the value of \texttt{new\_s} is
"string2" but \texttt{s} is left unchanged.

As mentioned in Chapter 1, strings can be compared with string
\index{comparison operator!string}comparison operators such as
\texttt{==}.

\iconcode{
if line[1] == "\#" then ...}

If you find you are writing many such tests, the string processing you
are doing may be more cleanly handled using the string scanning
facilities, described below. But first, here is some more detail on the
character set data type, which is used in many of the string scanning
functions.

\subsection{Character Sets}

A cset is a set of characters. It has the usual properties of sets:
order is not significant, and a character can only occur once in a
cset. A \index{cset literal}cset literal is represented with single
quotes:

\iconcode{
c := 'aeiou'}

Since characters can only occur once in a cset, duplicates in a cset
literal are ignored; for example,
\texttt{'aaiiee'} is equivalent to
\texttt{'aie'}. Strings can be
converted to csets and vice versa. Since csets do not contain
duplicates, when a string is converted to a cset, all the duplicates
are removed.

Therefore to see if a string is composed of all the vowels and no
consonants:

\iconcode{
if cset(s) == 'aeiou' then ...}

Or, to find the number of distinct characters in a string:

\iconcode{
n := *cset(s)}

The \texttt{!} operator generates the members of a cset in sorted order;
this is also useful in some situations.

\subsection{Character Escapes}

Both strings and csets rely on the backslash as an escape character
within string literals. A backslash followed by an \index{escape
codes}\textit{escape code} of one or more characters specifies a
non-printable or control character. Escape codes may be specified by a
numeric value given in hex or octal format - for example,
\texttt{"{\textbackslash}x41"}.
Alternatively, any control character may be specified with an escape
code consisting of the caret (\texttt{\^{}}) followed by the alphabetic
letter of the control character. A cset containing control-C,
control-D, and control-Z could be specified as
\texttt{'{\textbackslash}\^{}c{\textbackslash}\^{}d{\textbackslash}\^{}z'}.
For the most common character escapes, a single-letter code is defined,
such as \texttt{"{\textbackslash}t"} for
the tab character, or
\texttt{"{\textbackslash}n"} for the
newline. For all other characters, the character following the
backslash is the character; this is how quotes or backslashes are
included in literals. The escape codes are summarized in Table 3-1.

\medskip
% \pagebreak

\begin{center}
{\sffamily\bfseries
Table 3-1

Escape Codes and Characters 
}
\end{center}

\begin{center}
\begin{xtabular}{|m{0.38in}|m{0.8in}|m{0.4in}|m{0.855in}|m{0.38in}|m{1.14in}|m{0.38in}|m{0.68in}|}
\hline
\sffamily\bfseries Code &
\sffamily\bfseries Character &
\sffamily\bfseries Code &
\sffamily\bfseries Character &
\sffamily\bfseries Code &
\sffamily\bfseries Character &
\sffamily\bfseries Code &
\sffamily\bfseries Character\\\hline
\ \ {\textbackslash}b &
backspace &
\ \ {\textbackslash}d &
delete &
\ \ {\textbackslash}e &
escape &
\ \ {\textbackslash}f &
form feed\\\hline
\ \ {\textbackslash}l &
line feed &
\ \ {\textbackslash}n &
newline &
\ \ {\textbackslash}r &
carriage return &
\ \ {\textbackslash}t &
tab\\\hline
\ \ {\textbackslash}v &
vertical tab &
\ {\textbackslash}' &
quote &
\ \ {\textbackslash}" &
double quote &
\ \ {\textbackslash}{\textbackslash} &
backslash\\\hline
\ {\textbackslash}\textit{ooo} &
octal &
{\textbackslash}x\textit{hh} &
hexadecimal  &
\ {\textbackslash}\^{}\textit{x} &
Control-\textit{x} &
~
 &
~
\\\hline
\end{xtabular}
\end{center}

\section{String Scanning}

Strings are ordered sequences of symbols. A string's vital information
is conveyed both in its individual elements and in
the number and order in which the symbols appear.
There is a fundamental duality between writing the code to analyze a
string, and writing down some data that describes or abstracts
that string. The same duality is seen in Unicon's string scanning
control structure described in this section, and the pattern data type
used in matching operators, which is described in this next section.

Unicon's main building block for string analysis is a control
structure called \index{string!scanning}\textit{string scanning}. A
\index{scanning!environment}scanning environment consists of a string
\index{subject string}\index{string!subject}\textit{subject} and an
integer \index{position,
string}\index{string!position}\textit{position} within the subject at
which scanning is to be performed. These values are held by the keyword
variables \texttt{\&subject} and \texttt{\&pos}. Scanning environments
are created by an expression of the form

\iconcode{
\textit{s} ? \textit{expr}}

The binary \texttt{?} operator sets the subject to its left argument and
initializes the position to 1; then it executes the expression on the
right side.

The expression usually has an interesting combination of various
\index{matching functions}\textit{matching
functions} in it. Matching functions change the position, and return
the substring between the old and new positions. For example:
\texttt{move(j)} moves the position \texttt{j} places to the
right and returns the substring between the old and new position. This
string will have exactly \texttt{j} characters in it. When the position
cannot move as directed, for example because there are less than
\texttt{j} characters to the right, \index{move(i)}\texttt{move()}
fails. Here is a simple example:

\iconcode{
text ? \{ \\
\>   while move(1) do \\
\>   \ \ \ write(move(1)) \\
\>   \}
}

This code writes out every other character of the string in variable
\texttt{text}.

Another function is \index{tab(i)}\texttt{tab(i)}, which sets the
position \texttt{\&pos} to its argument and returns the substring that
it passed over. So the expression \texttt{tab(0)} will return the
substring from the current position to the end of the string, and set
the position to the end of the string.

Several string scanning functions examine a string and generate the interesting
positions in it. We have already seen \index{find()}\texttt{find()},
which looks for substrings. In addition to the other parameters that
define what the function looks for, these string functions end with
three optional parameters: a string to examine and two integers. These
functions \index{default!scanning parameters}default their string
parameter to \texttt{\&subject}, the string being scanned. The two
integer positions specify where in the string the processing will be
performed; they default to 1 and 0 (the entire string), or
\texttt{\&pos} and 0 if the string defaulted to \texttt{\&subject}.
Here is a \index{generator}generator that produces the words from the
input:

\iconcode{
procedure getword() \\
local wchar, line \\
\>   wchar := \&letters ++ \&digits ++
'{\textbackslash}'-' \\
\>   while line := read() do \\
\>   \ \ \ line ? while tab(upto(wchar)) do \{ \\
\>   \ \ \ \ \ \ word := tab(many(wchar)) \\
\>   \ \ \ \ \ \ suspend word \\
\>   \ \ \ \ \ \ \} \\
end
}

Variable \texttt{wchar} is a cset of characters that are
allowed in words, including apostrophe (which is escaped) and hyphen
characters. \index{upto(c)}\texttt{upto(c)} returns the next position
at which a character from the cset \texttt{c} occurs. The
\index{many(c)}\texttt{many(c)} function returns the position after a
sequence of characters from \texttt{c}, if one or more of them occur at
the current position. The expression \texttt{tab(upto(wchar))} advances
the position to a character from \texttt{wchar}; then
\texttt{tab(many(wchar))} moves the position to the end of the word and
returns the word that is found. This is a generator, so when it is
resumed, it takes up execution from where it left off and continues to
look for words (reading the input as necessary).

Notice the first line: the cset \texttt{wchar} is the set union of the
upper- and lowercase letters (the value of the keyword
\texttt{\&letters}) and the digits (the keyword \texttt{\&digits}).
This cset union is performed each time \texttt{getword()} is called,
which is inefficient if \texttt{getword()} is called many times. The
procedure could instead calculate the value once and store it for all
future calls to \texttt{getword()}.

Declaring the variable to be static will cause its value to
persist across calls to the procedure. Normal local variables are
initialized to the null value each time a procedure is entered. To do
this, add these two lines to the beginning of the procedure:

\iconcode{
static wchar \\
initial wchar := \&letters ++ \&digits ++
'{\textbackslash}'-'
}

The \index{match(s)}\texttt{match(s)} function takes a string argument
and succeeds if \texttt{s} is found at the current position in the
subject. If it succeeds, it produces the position at the end of the
matched substring. This expression

\iconcode{
if tab(match("-")) then sign := -1 else sign
:= 1}

\noindent
looks to see if there is a minus sign at the current position; if one is
found, \texttt{\&pos} is moved past it and the variable \texttt{sign}
is assigned a -1; otherwise, it gets a 1. The expression
\texttt{tab(match(s))} occurs quite often in string scanning, so it is
given a shortcut: \texttt{=s}.  The section on pattern matching
later in this chapter will explain that this
``unary equals'' operator has an additional, more powerful use.

The last two string scanning functions to round out
Icon's built-in repertoire are
\index{any(c)}\texttt{any(c)} and \index{bal()}\texttt{bal(c1,c2,c3)}.
\texttt{any(c)} is similar to \texttt{many()}, but only tests a single
character being scanned to see if it is in cset \texttt{c}. The
\texttt{bal()} function produces positions at which a character in
\texttt{c1} occurs, similar to \texttt{upto()}, with the added
stipulation that the string up to those positions is \textit{balanced}
with respect to characters in \texttt{c2} and \texttt{c3}. A string is
balanced if it has the same number of characters from \texttt{c2} as
from \texttt{c3} and there are at no point more \texttt{c3} characters
present than \texttt{c2} characters. The \texttt{c1} argument defaults
to \texttt{\&cset}. Since \texttt{c2} and \texttt{c3} default to
\texttt{'('} and
\texttt{')'}, \texttt{bal()} defaults
to find balanced parentheses.

The restriction that \texttt{bal()} only returns positions at which a
character in \texttt{c1} occurs is a bit strange. Consider what you
would need to do in order to write an expression that tells whether a
string \texttt{s} is balanced or not.

You might want to write it as \texttt{s ? (bal() = *s+1)} but
\texttt{bal()} will never return that position. Concatenating an extra
character solves this problem:

\iconcode{
procedure isbalanced(s) \\
\>   return (s {\textbar}{\textbar} " ") ?
(bal() = *s+1) \\
end
}

If string \texttt{s} is very large, this solution is not cheap, since it
creates a new copy of string \texttt{s}. You might write a version of
\texttt{isbalanced()} that doesn't use the
\texttt{bal()} function, and see if you can make it run faster than
this version. An example later in this chapter shows how to use
\texttt{bal()} in a more elegant manner.

\subsection*{File Completion}

Consider the following gem, attributed to Jerry \index{Nowlin,
Jerry}Nowlin and Bob \index{Alexander, Bob}Alexander. Suppose you want
to obtain the full name of a file, given only the first few letters of
a filename and a list of complete \index{filename completion}filenames.
The following one line procedure does the trick:

\iconcode{
procedure complete(prefix, filenames) \\
\>   suspend match(prefix, p := !filenames) \& p \\
end
}

This procedure works fine for lists with just a few members and also for
cases where \texttt{prefix} is fairly large.

\subsubsection*{Backtracking}

\index{backtracking}The matching functions we have seen so far,
(\texttt{tab()} and \texttt{move()}), are actually
\index{generator}generators. That is, even though they only produce one
value, they suspend instead of returning. If expression evaluation ever
resumes one of these functions, they restore the old value of
\texttt{\&pos}. This makes it easy to try alternative matches starting
from the same position in the string:

\iconcode{
s ? (="0x" \& tab(many(\&digits ++
'abcdefABCDEF'))) {\textbar} \\
\>   tab(many(\&digits))
}

This expression will match either a hexadecimal string in the format
used by C or a decimal integer. Suppose \texttt{s} contains the string
\texttt{"0xy"}. The first part of the
expression succeeds and matches the
\texttt{"0x"}; but then the expression
\texttt{tab(many(\&digits ++
'abcdef'))} fails; this causes Unicon
to resume the first \texttt{tab()}, which resets the position to the
beginning of the string and fails. Unicon then evaluates the expression
\texttt{tab(many(\&digits))} which succeeds (matching the string
\texttt{"0"}); therefore the entire
expression succeeds and leaves \texttt{\&pos} at 2.

{\sffamily\bfseries
Warning}

Be careful when using \texttt{tab()} or \texttt{move()} in a
surrounding expression that can fail! The fact that \texttt{tab()}
and \texttt{move()} reset \texttt{\&pos} upon expression
failure causes confusion and bugs when it happens accidentally.

\subsection*{Concordance Example}

Listing 3-1 illustrates the above concepts and introduces a few more.
Here is a program to read a file, and generate a
\index{concordance}concordance that prints each word followed by a list
of the lines on which it occurs. Short words like
\texttt{"the"} aren't
interesting, so the program only counts words longer than three
characters. 

\bigskip

{\sffamily\bfseries Listing 3-1}
{\sffamily\bfseries A simple concordance program}

\iconcode{
procedure main(args) \\
\>   (*args = 1) {\textbar} stop("Need a
file!") \\
\>   f := open(args[1]) {\textbar}
stop("Couldn't open ",
args[1]) \\
\>   wordlist := table() \\
\>   lineno := 0 \\
\ \\
\>   while line := map(read(f)) do \{ \\
\>   \ \ \ lineno +:= 1 \\
\>   \ \ \ every word := getword(line) do  \\
\>   \ \ \ \ \ \ if *word {\textgreater} 3 then \{ \\
\>   \ \ \ \ \ \ \ \ \ \# if word isn't in the table, 
	set entry to empty list \\
\>   \ \ \ \ \ \ \ \ \ /wordlist[word] := list() \\
\>   \ \ \ \ \ \ \ \ \ put(wordlist[word], lineno) \\
\>   \ \ \ \ \ \ \ \ \ \} \\
\>   \ \ \ \} \\
\>   L := sort(wordlist) \\
\>   every l := !L do \{ \\
\>   \ \ \ writes(l[1],
"{\textbackslash}t") \\
\>   \ \ \ linelist := "" \\
\>   \ \ \ \# Collect line numbers into a string \\
\>   \ \ \ every linelist {\textbar}{\textbar}:= (!l[2]
{\textbar}{\textbar} ", ") \\
\>   \ \ \ \# trim the final ", " \\
\>   \ \ \ write(linelist[1:-2]) \\
\>   \ \ \ \} \\
end \\
\ \\
procedure getword(s) \\
\>   s ? while tab(upto(\&letters)) do \{ \\
\>   \ \ \ word := tab(many(\&letters)) \\
\>   \ \ \ suspend word \\
\>   \ \ \ \} \\
end
}

\noindent If we run this program on this input: 

\iconcode{
Half a league, half a league, \\
Half a league onward, \\
All in the valley of Death \\
Rode the six hundred.
}

\noindent the program writes this output: 

\iconcode{
death \ \ 3 \\
half \ \ \ 1, 2 \\
hundred 4 \\
league \ 1, 1, 2 \\
onward \ 2 \\
rode \ \ \ 4 \\
valley \ 3
}

First, note that the \texttt{main()} procedure requires a command-line
argument, the name of a file to open. Also, we pass all the lines read
through the function \texttt{map()}. This is a function that takes
three arguments, the first being the string to map; and the second and
third specifying how the string should be mapped on a character by
character basis. The defaults for the second and third arguments are
the uppercase letters and the lowercase letters, respectively;
therefore, the call to \texttt{map()} converts the line just read in to
all \index{lower case}lowercase.

\section{Pattern Matching}

Pattern matching in Unicon is like string scanning on steroids.
Patterns encode as data what sort of strings to match, instead of writing
string scanning code to perform the match directly.  A
pattern data type allows complex patterns to be composed from pieces.
The patterns can then be used in the middle of string scans to give
that notation a boost, or used on their own. Arguably, you don't need
patterns because anything that can be done to strings, can be done
using string scanning. But when the pattern solution is usually
shorter, more readable, and runs faster, why wouldn't everyone use them?

Patterns are understood in terms of two different points in time, the
point when the pattern is constructed, and the times at which it is
used to match a string.  Most of the programming work for patterns
involves formulating the pattern construction, but most of the
computation occurs later on during the pattern matches.
The next two subsections describe ways to
create many and various complex patterns, while the two notations for
using patterns are relatively simple and require little space.  All of
this only becomes clear with numerous examples that will follow.


\subsection{Regular Expressions}

The literal values of the pattern type are regular expressions,
enclosed in less than ($<$) and greater than ($>$) symbols. The
notation of regular expressions is very old and very famous in
computer science, and readers already familiar with them may wish
to skim this section.

Within $<$ and $>$ symbols, the normal Unicon interpretation of
operators does not apply; instead a set of regular expression
operators is used to express simple string patterns concisely.  The
following are examples of regular expressions.

\bigskip

\begin{tabular}{|l|l|} \hline
regular expression	& is a pattern that... \\ \hline
\texttt{$<$abc$>$}	& matches abc \\
\texttt{$<$a$|$b$|$c$>$}& matches a or b or c \\
\texttt{$<$[a-c]$>$}	& matches a or b or c \\
\texttt{$<$ab?c$>$}	& matches a followed optionally by b followed by c \\
\texttt{$<$ab*c$>$}	& matches a followed by 0 or more b's followed by c \\
\texttt{$<$a*b*c*$>$}	& matches a's followed by b's followed by c's \\ \hline
\end{tabular}

\subsection{Pattern Composition}

Regular expressions are awesome, but there are many patterns that they
cannot express. Unicon has many pattern functions and operators that
construct new patterns, often from existing pattern
arguments. Sometimes, they simply make it convenient to store parts of
patterns in variables that can then be used at various places in
larger patterns. Other times, they make it possible to write patterns
are not easily written as regular expressions.

\bigskip

\begin{tabular}{|l|l|} \hline
composer & constructs a pattern that... \\ \hline
\texttt{p1 {\textbar\textbar} p2} & matches if pattern p1 is followed by pattern p2 \\
\texttt{p1 .{\textbar} p2} & matches if pattern p1 or pattern p2 matches \\
\texttt{p -$>$ v}   & assigns s to v if an entire pattern match succeeds \\
\texttt{p =$>$ v}   & assigns s to v if a pattern match makes it here \\
\texttt{.$>$ v}   & assigns the current position in the match to v \\
\texttt{`v`}      & evaluates v at pattern match time \\
\texttt{Abort()}  & causes an entire pattern match to fail immediately \\
\texttt{Any(c)}   & matches a character in cset c \\
\texttt{Arb()}    & matches anything \\
\texttt{Arbno(p)} & matches pattern p as few (zero or more) times as possible \\
\texttt{Bal()}    & matches the shortest non-null substring with balanced parentheses\\
\texttt{Break(c)} & matches the substring until a character in cset c occurs \\
\texttt{Breakx(c)}& matches the substring until a character in cset c occurs \\
\texttt{Fail()}   & fails to match, triggering any alternative(s) \\
\texttt{Fence()}  & fails to match, preventing alternative(s) \\
\texttt{Len(i)}   & matches any i characters \\
\texttt{NotAny(c)} & matches any one character not in cset c \\
\texttt{Nspan(c)} & matches 0 or more characters in cset c, as many as possible \\
\texttt{Pos(i)}   & sets the cursor to position i \\
\texttt{Rem()}    & matches the remainder of the string \\
\texttt{Span(c)}  & matches 1 or more characters in cset c, as many as possible \\
\texttt{Succeed()}& causes the entire pattern match to succeed immediately \\
\texttt{Tab(i)}   & matches from the current position to position i, moving to
	   that location \\
\texttt{Rpos(i)}  & sets the position i characters from the right end of the
	   string \\
\texttt{Rtab(i)}  & matches from the current position to i
	   characters from the right end. \\ \hline
\end{tabular}

\bigskip

This table summarizes a facility for which an entire chapter could be
written. Besides what extra information you find on these functions in
this chapter and in Appendix A, Unicon Technical Report 18 covers
these constructors in more detail. The concepts generally are
translated directly from SNOBOL4, so consulting SNOBOL4 books may also
be of use.

Most operands and arguments are required to be of type pattern, with the
exception of those marked as type integer (i) or cset (c), and those
which are variables (v). If a pattern is required, a cset may be
supplied, with semantics equivalent to the pattern which will match
any member of the cset. Otherwise if the argument is not a pattern
it will be converted to a string; strings are converted to patterns
that match the string.

Variable operands may be simple variables or references with a
subscript or field operator.  The translator may not currently handle
arbitrarily complex variable references within patterns.  The
unevaluated expression (backquotes) operator does handle
function calls and simple method invocations in addition to
variables.


\subsection{Pattern Match Operators}

A pattern match is performed within a string scanning environment
using the unary equals operator, \texttt{=p}.  If \texttt{p}
is matched at the current position, \texttt{=p} produces the
substring matched and moves the position by that amount.

There is also a pattern match control structure, \texttt{s ?? p}, which creates
a new string scanning environment for \texttt{s} and looks for pattern
\texttt{p} within \texttt{s}, working from left to right.

\subsection{Scopes of Unevaluated Variables}

Since a pattern can be passed as a parameter, variables used in
patterns might get used outside of the scope where the pattern was
constructed, potentially anywhere in the program. In SNOBOL4 this was
not an issue mostly because all variables were global.  In Unicon
variables are not global by default, and the variables used
during pattern matching are evaluated in the scope of the pattern match,
not references to locals that existed back during pattern construction
time.

To make things more fun, it is impractical to apply the usual rules
for implicit variable declaration to variables that do not appear in a
procedure body because they are referenced in a pattern that was
constructed elsewhere.  If you use a variable in a pattern and pass
that pattern into a different scope, you must declare that variable
explicitly, either as a global or in the scope where it is used in a
pattern match.

\section{String Scanning and Pattern Matching Miscellany}

Many topics related to string scanning and pattern matching do not
easily fit into one of the preceding sections, but are nevertheless
important.

\subsection{Grep}

Grep, an acronym defined variously, is one of the oldest UNIX
utilities, which searches files for occurrences of a pattern defined
by a regular expression.

\bigskip

{\sffamily\bfseries Listing 3-2}
{\sffamily\bfseries A simple grep-like program}

\iconcode{
link regexp \\
procedure main(av) \\
\>   local f, re, repl \\
\>   every (f{\textbar}re{\textbar}repl) := pop(av) \\
\>   f := open(f) {\textbar} stop("can't open file named: ", f) \\
\>   while line := read(f) do \\
\>   \ \ \ write(re\_sub(line, re, repl)) \\
end \\
procedure re\_sub(str, re, repl) \\
\>   result := "" \\
\>   str ? \{ \\
\>   \ \ \ while j := ReFind(re) do \{ \\
\>   \ \ \ \ \ \ result {\textbar}{\textbar}:= tab(j)
{\textbar}{\textbar} repl \\
\>   \ \ \ \ \ \ tab(ReMatch(re)) \\
\>   \ \ \ \ \ \ \} \\
\>   \ \ \ result {\textbar}{\textbar}:= tab(0) \\
\>   \ \ \ \} \\
\>   return result \\
end
}

To replace all occurrences of
\texttt{"read{\textbar}write"} with
\texttt{"IO operation"} you could type 

\iconcode{
igrep mypaper.txt "read{\textbar}write"
"IO Operation"}

Since the program has access to the pattern matching operation at a
finer grain, more complex operations are possible, this
search-and-replace is just an example.

\subsection{Grammars}

\index{grammar}Grammars are collections of rules that describe
\index{syntax}\textit{syntax}, the combinations of words allowed in a
language. Grammars are used heavily both in linguistics and in computer
science. \index{pattern matching}Pattern matching using a grammar is
often called \index{parse}\textit{parsing}, and is one way to match
patterns more complex than regular expressions can handle. This section
presents some simple programming techniques for parsing context free
grammars. Context free grammars utilize a \index{stack}stack to
recognize a fundamentally more complex category of patterns than
regular expressions can; they are defined below.

For linguists, this treatment is elementary, but introduces useful
programming techniques.
%% Chapter 18 refers to lexyacc.tex, which is no longer in the book.
%% For writers of programming language compilers,
%% an automatic parser generator tool that you can use with Unicon or Icon
%% is described in Chapter 18.
If you are not interested in grammars, you
can skip the rest of this chapter.

A \index{context-free grammar}context-free grammar or CFG is a set of
rules or \textit{productions}. Here is an example:

\iconcode{
S -{\textgreater} S S \\
\>   {\textbar} ( S ) \\
\>   {\textbar} ( )
}

This grammar has three productions. There are two kinds of symbols,
\textit{non-terminals} like \texttt{S} that can be replaced by the
string on the right side of a rule, and \textit{terminals} like
\texttt{(} and \texttt{)}. An application of a production rule is called
a derivation. One special non-terminal is called the
\textit{start symbol}; a string is accepted by the grammar if there is a
sequence of derivations from the start symbol that leads to the string.
By convention the start symbol is the first non-terminal in the
definition of the grammar. (This grammar only has one non-terminal, and
it is also the start symbol.)

This grammar matches all strings of balanced parentheses. The string
\texttt{(()(()()))} can be matched by this derivation:

\iconcode{
S -{\textgreater} (S) -{\textgreater} (SS) -{\textgreater} (()S)
-{\textgreater} (()(S)) -{\textgreater} \\
\>   \ \ (()(SS)) -{\textgreater} (()(()S)) -{\textgreater} (()(()()))
}

\subsection*{Parsing}

This section is a discussion of parsers written by hand in Unicon.
It would not be right to talk about parsing context free grammars
without mentioning the standard tool, iyacc, that the Unicon language
translator itself is written in.  Iyacc is an industrial strength
parser generator, derived from the open source "Berkeley yacc", that
generates parsers as .icn source files compatible with Icon and Unicon.
Iyacc comes with Unicon distributions and is documented in Unicon Technical
Report 3 at http://unicon.org/utr/utr3.pdf. 

Unicon can parse grammars in a natural way using matching
functions. A production

\iconcode{
A -{\textgreater} B a D  \\
\>   {\textbar} C E b
}

\noindent
can be mapped to this matching function:

\iconcode{
procedure A() \\
\>   suspend (B() \& ="a" \& D())
{\textbar} (C() \& E() \& ="b") \\
end
}

\noindent
This procedure first tries to match a string matched by \texttt{B},
followed the character \texttt{a}, followed by a string matched by
\texttt{D}. If \texttt{D} \index{expression failure}fails, execution
backtracks across the \texttt{="a"}
(resetting \texttt{\&pos}) and resume \texttt{B()}, which will attempt
the next match.

If the sub-expression to the left of the \index{alternation operator (
{\textbar} )}alternation fails, then execution will try the
sub-expression on the right, \texttt{C() \& E() \&
="b"} until something matches - in which
case \texttt{A} succeeds, or nothing matches - which will cause it to
fail.

Parsers for any CFG can be written in this way. However, this is an
expensive way to do it! Unicon's
expression evaluation will try all possible derivations
trying to match a string. This is not a good way to parse, especially
if the grammar is amenable to lookahead methods.
%% The chapter 18 referred to here is lexyacc.tex (which was removed
%% in January 2013).
%% A more efficient
%% method is given in the next section. For serious parsing jobs, Chapter
%% 18 shows how to use the Unicon versions of the standard
%% industrial-strength lexical analyzer and parser generation tools, lex
%% and yacc.

\subsection*{Doing It Better}

Many grammars can be parsed more efficiently using
well-known techniques - consult a book on compilers for details. Here
is one way of parsing a grammar using some of the built-in
functions. Consider this grammar for an arithmetic expression:

\iconcode{
E -{\textgreater} T {\textbar} T + E \\
T -{\textgreater} F {\textbar} F * T \\
F -{\textgreater} a {\textbar} b {\textbar} c {\textbar} ( E )
}

\noindent
Listing 3-3 is an Unicon program that recognizes strings produced by
this grammar:

\bigskip

{\sffamily\bfseries Listing 3-3}
{\sffamily\bfseries Expression parser}

\iconcode{
procedure main() \\
\>   while line := read() do \\
\>   \ \ \ if expr(line) == line then write("Success!") \\
\>   \ \ \ else write("Failure.") \\
end \\
procedure expr(s) \\
\>   s ? \{ \\
\>   \ \ \ while t := tab(bal('+'))
do \{ \\
\>   \ \ \ \ \ \ term(t) {\textbar} \index{fail}fail ;
="+" \\
\>   \ \ \ \ \ \ \} \\
\>   \ \ \ term(tab(0)) {\textbar} fail \\
\>   \ \ \ \} \\
\>   return s \\
end \\
procedure term(s) \\
\>   s ? \{ \\
\>   \ \ \ while f := tab(bal('*'))
do \{ \\
\>   \ \ \ \ \ \ factor(f) {\textbar} fail ;
="*" \\
\>   \ \ \ \ \ \ \} \\
\>   \ \ \ factor(tab(0)) {\textbar} fail \\
\>   \ \ \ \} \\
\>   return s \\
end \\
procedure factor(s) \\
\>   s ? suspend ="a" {\textbar}
="b" {\textbar}
="c" {\textbar}  ( ="("
{\textbar}{\textbar}
expr(tab(bal(')')))
{\textbar}{\textbar} =")" ) \\
end
}

The interesting procedure here is \index{bal()}\texttt{bal()}. With
\texttt{')'} as its first argument,
\texttt{bal()} scans to the closing parenthesis, skipping over any
parentheses in nested subexpressions, which is exactly what is needed
here.

The procedure \texttt{factor()} is written according to the rule in the
previous section. The procedures \texttt{expr()} and \texttt{term()}
have the same structure. The \texttt{expr()} procedure skips any
subexpressions (with balanced parentheses) and looks for a \texttt{+}.
We know that this substring is a well-formed expression that is not a
sum of terms, therefore, it must be a term. Similarly \texttt{term()}
looks for \texttt{*} and it knows that the expression does not contain
any \texttt{*} operators at the same nesting level; therefore it must
be a factor.

Notice that the procedures return the strings that they matched. This
allows us to check if the whole line matched the grammar rather than
just an initial substring. Also, notice that \texttt{factor()} uses
string concatenation instead of conjunction, so that it can return the
matched substring.

\section*{Summary}

Unicon's string processing facilities are extensive.
Simple operations are very easy, while more complex string analysis has
the support of a special control structure, string scanning. String
scanning is not as concise as regular expression pattern matching, but
it is fundamentally more general because the code and patterns are
freely intermixed.


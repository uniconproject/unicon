\clearpage\section{Chapter 8: The Object-Oriented Way of Thinking}
\index{object-oriented programming}Object-oriented programming means
different things to different people. In Unicon, object-oriented
programming starts with encapsulation, inheritance, and polymorphism.
These ideas are found in most object-oriented languages as well as many
languages that are not object-oriented. The next several chapters
present these ideas and illustrate their use in design diagrams and
actual code. Both diagrams and code are relevant because they are
alternative notations by which programmers share their knowledge. This
chapter explores the essence of object-orientation and gives you the
concepts needed before you delve into diagrams and code examples. In
this chapter you will learn:

\begin{itemize}
\item How different programming languages support objects in different
ways
\item To simplify programs by encapsulating data and code
\item The relationship between objects and program design
\end{itemize}
\subsection{Objects in Programming Languages}
Object-oriented programming can be done in any language, but some
languages make it easier than others. Support for objects should not
come at the cost of strange syntax or programs that look funny in a
heterogeneous desktop-computing environment. \index{Smalltalk}Smalltalk
has these problems. C++ is based on a well-known syntax and is used for
mainstream desktop computing applications, but its low-level
machine-orientation makes C++ less than ideal as an algorithmic
notation usable by non-experts. \index{Java}Java offers a simple object
model and familiar syntax - a potent combination. The main advantages
Unicon has over Java are fundamentally higher-level built-in types,
operations, and control structures.

Many object-oriented languages require that \textit{everything} be done
in terms of objects, even when objects are not appropriate. Unicon
provides objects as just another tool to aid in the writing of
programs, especially large ones. Icon already provides a powerful
notation for expressing a general class of algorithms. The purpose of
object-orientation is to enhance that notation, not to get in the way
of it.

Icon does not support user-defined objects, although its built-in types
have nice object-like encapsulation and polymorphism properties.
Unicon{\textquotesingle}s object-oriented facilities descend from a
package for Icon called \index{Idol}Idol. In Idol, a preprocessor
implemented objects with no support from the underlying Icon runtime
system. In contrast, Unicon has support for objects built-in to the
language. This simplifies the notation and improves the performance of
object-related computations.

Object-orientation adds several general concepts into procedure-based
programming. The single overriding reason for \index{object-oriented
programming}object-oriented programming is to reduce complexity in
large programs. Simple programs can be written easily in any language.
Somewhere between the 1,000-line mark and the 10,000-line mark most
programmers can no longer keep track of their entire program at once.
By using a very high-level programming language, fewer lines of code
are required; a programmer can write perhaps ten times as large a
program and still be able to keep track of things.

As programmers write larger and larger programs, the benefit provided by
\index{very high-level language}very high-level languages does not keep
up with program complexity. This obstacle has been labeled the
{\textquotedbl}software crisis,{\textquotedbl} and object-oriented
programming is one way to address this crisis. In short, the goals of
object-oriented programming are to reduce the complexity of coding
required to write very large programs and to allow code to be
understood independently of the context of the surrounding program. The
techniques employed to achieve these goals are discussed below. 

A second reason to consider object-oriented programming is that the
paradigm fits certain problem domains especially well, such as
simulation, and graphical user interfaces. The first well-known
object-oriented language, \index{Simula67}Simula67, certainly had the
domain of simulation in mind. The second pioneering object-oriented
language, Smalltalk, popularized fundamental aspects of bitmapped
graphical user interfaces that are nearly universal today. Three
decades of experience with object-oriented techniques has led many
practitioners to conclude that the concepts presented below are very
general and widely applicable, but not all problems fit the
object-oriented mold. Unicon advocates the use of objects, but this is
a suggestion, not a rule. 

\subsubsection{Encapsulation}
The primary concept advocated by object-oriented programming is the
principle of \index{encapsulation}encapsulation. Encapsulation is the
isolation, in the source code that a programmer writes, of a data
representation and the code that manipulates the data representation.
In some sense, encapsulation is an assertion that no other routines in
the program have {\textquotedbl}side-effects{\textquotedbl} with
respect to the data structure in question. It is easier to reason about
encapsulated data because all of the source code that could affect that
data is immediately present with its definition. 

Encapsulation does for data structures what the procedure does for
algorithms: it draws a line of demarcation in the program source code.
Code outside this boundary is (or can be, or ought to be) irrelevant to
the code that is inside, and vice versa. All communication across the
boundary occurs through a public interface. We call an encapsulated
data structure an object. Just as a set of named variables called
parameters comprises the only interface between a procedure and the
code that uses it, a set of named procedures called methods comprises
the only interface between an object and the code that uses it. 

This textual definition of encapsulation as a property of program source
code accounts for the fact that good programmers can write encapsulated
data structures in any language. The problem is not capability, but
verification. To verify encapsulation some languages require
programmers to specify the visibility of every piece of information in
each data structure as {\textquotedbl}public{\textquotedbl} or
{\textquotedbl}private.{\textquotedbl} There are even multiple forms of
privacy (semi-private?). Unicon instead stresses simplicity. 

\subsubsection{Inheritance}
In large programs, the same or nearly the same data structures are used
over and over again for myriad different purposes. Similarly,
variations on the same algorithms are employed by structure after
structure. To minimize redundancy, techniques are needed to support
code sharing for both data structures and algorithms. Code is shared by
related data structures through a programming concept called
\index{inheritance}inheritance. 

The basic premise of inheritance is simple: when writing code for a data
structure that is similar to a structure that is already written, one
specifies the new structure by giving the differences between it and
the old structure, instead of copying and then modifying the old
structure{\textquotesingle}s code. Obviously there are times when the
inheritance mechanism is not useful, such as if the two data structures
are more different than they are similar, or if they are simple enough
that inheritance would only confuse things, for example. 

Inheritance addresses a variety of common programming problems found at
different conceptual levels. The most obvious software engineering
problem it solves might be termed enhancement. During the development
of a program, its data structures may require extension via new state
variables or new operations or both; inheritance is especially useful
when both the original structure and the extension are used by the
application. Inheritance also supports simplification, or the reduction
of a data structure{\textquotesingle}s state variables or operations.
Simplification is analogous to \index{argument culling}\textit{argument
culling}\textit{,} an idea from lambda calculus (don{\textquotesingle}t
worry if this sounds like Greek to you), in that it describes a logical
relation between structures. In general, inheritance may be used in
source code to describe any sort of relational hyponymy, or special
casing. In Unicon the collection of all inheritance relations defines a
directed (not necessarily acyclic) graph. 

\subsubsection[Polymorphism]{Polymorphism}
\index{polymorphism}From the perspective of the writer of related data
structures, inheritance provides a convenient method for code sharing,
but what about the code that uses objects? Since objects are
encapsulated, that code is not dependent upon the internals of the
object at all, and it makes no difference to the client code whether
the object in question belongs to the original class or the inheriting
class. 

In fact, we can make a stronger statement. Due to encapsulation,
different executions of some code that uses objects to implement a
particular algorithm may operate on different objects that are not
related by inheritance at all. Such code can utilize any objects that
implement the operations that the code invokes. This facility is called
polymorphism, and such algorithms are called generic. This feature is
found in many non-object-oriented languages; in object-oriented
languages it is a natural extension of encapsulation. 

\subsection[Objects in Program Design]{Objects in Program Design}
\index{program design}Another fundamental way to think about objects is
from the point of view of software design. During program design,
objects are used to model the problem domain. The different kinds of
objects and relationships between objects capture fundamental
information that organizes the rest of the program{\textquotesingle}s
design and implementation. Program design includes several other
fundamental tasks such as the design of the user interface, or
interactions with external systems across a network. Additional kinds
of modeling are used for these tasks, but they all revolve around the
object model.

The Unified Modeling Language (\index{UML}UML) is a diagramming notation
for building software models. It was invented by Grady Booch, Ivar
Jacobson, and James Rumbaugh of Rational Software. In UML, software
models document the purpose and method of a software system. For small,
simple, or well-understood software projects, a prose description may
be all the documentation that is needed. UML is of interest to
designers of larger software systems for which a prose description
alone would be inadequate. The advantage of a model is that it conveys
information that is both more precise and more readily understood than
a prose description. UML is used during multiple phases of the software
lifecycle. UML defines several kinds of diagrams, of which we will only
consider four in this book.

\begin{itemize}
\item \textbf{Use case diagrams} show the organization of the
application around the specific tasks accomplished by different users.
\item \textbf{Class diagrams} show much of the static structure of the
application data.
\item \textbf{Statechart diagrams} model dynamic behavior of systems
that respond to external events, including user input.
\item \textbf{Collaboration diagrams} model interactions between
multiple objects
\end{itemize}
These diagrams describe key aspects of many categories of software
applications. The reader should consult the UML Notation Guide and
Semantics documents for a complete description of UML. \ A good
introduction is given in \textit{UML Toolkit}, by Hans-Erik Eriksson
and Magnus Penker (1998).

A typical application of UML to a software development project would use
these kinds of diagrams in sequence. You might start by constructing
use case diagrams and detailed descriptions of the different kinds of
users and tasks performed using the system. Then develop class diagrams
that capture the relationships between different kinds of objects
manipulated by the system. Finally, construct statechart and
collaboration diagrams as needed to describe the sequences of events
that can occur and the corresponding operations performed by various
objects in response to those events.

While use case and statechart diagrams are important, their purpose is
to help organize and elaborate on the object model represented by one
or more class diagrams. For this reason, the next two chapters present
elements of class diagrams along with the corresponding programming
concepts. Use case diagrams, statecharts, and collaboration diagrams
are discussed in chapter 12.

{\sffamily
Summary}

Unicon provides object-oriented programming facilities as a collection
of tools to reduce the complexity of large \ programs. These tools are
encapsulation, inheritance, and polymorphism. Since a primary goal of
Unicon is to promote code sharing and reuse, various specific
programming problems have elegant solutions available in the Unicon
class library.


\bigskip

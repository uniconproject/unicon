\chapter{Virtual Machine Instructions}

\PrimaryIndex{Virtual machine!Instruction}
This appendix lists all the Icon and Unicon virtual machine instructions. For
instructions that correspond to source-language operators, only the
corresponding operations are shown. Unless otherwise specified,
references to the stack mean the interpreter stack.
If $j > i$ then $expr_j$ is closer to the top of the stack than $expr_i$
(i.e. $expr_j$ will be popped before $expr_i$).

\noindent\textcolor{blue}{Instructions coloured in blue have been added to the Unicon
virtual machine.}
\bigskip

\newcommand{\uniconop}[1]{\color{blue} \texttt{#1} & \textcolor{blue}{To be completed}}

\tablehead{}
\tablefirsthead{}
\tabletail{}
\tablelasttail{}

\begin{xtabular}{l@{\hspace{1.5cm}}p{13.5cm}}

\texttt{acset a} & Push a descriptor for the cset block at address
\texttt{a} onto the stack. \\

\texttt{aglobal a} & Push a variable descriptor pointing to global
identifier at address \texttt{a} onto the stack.\\

\texttt{agoto a} & Set \texttt{ipc} to address \texttt{a}.\\

\texttt{amark a}    & Create an expression frame whose marker contains
the failure
\texttt{ipc} corresponding to the address \texttt{a}, the current \texttt{efp},
\texttt{gfp}, and \texttt{ilevel}.\\

\texttt{apply}   & $expr_1$ \texttt{!} $expr_2$\\

\texttt{areal a} & Push a descriptor for the real number block at address
\texttt{a} onto the stack.\\

\texttt{arg n} & Push a variable descriptor pointing to argument \texttt{n}.\\

\texttt{asgn}  & $expr_1$ \texttt{:=} $expr_2$\\

\texttt{astatic a} & Push a variable descriptor pointing to static
identifier at address texttt{a}.\\

\texttt{astr n, a} & Push a descriptor for the string of length \texttt{n} 
at address \texttt{a}.\\

\texttt{bang}  & \texttt{!$expr$}\\

\texttt{bscan} & Push the current values of \texttt{\&subject} and
\texttt{\&pos}. Convert the descriptor prior to these two descriptors into a
string.  If the conversion cannot be performed, terminate execution with an
message. Otherwise, assign it to \texttt{\&subject} and assign 1 to
\texttt{\&pos}. Then suspend. If resumed, restore the former values of
\texttt{\&subject} and \texttt{\&pos} and fail.\\

\texttt{cat}   & \texttt{$expr_1$ || $expr_2$}\\

\texttt{ccase} & Push a copy of the descriptor just below the current expression frame.\\

\texttt{chfail n} &  Change the failure ipc in the current expression frame
marker to \texttt{n}.\\

\texttt{coact} & Save the current state information in the current co-expression
block, restore state information from the co-expression block being activated,
perform a context switch, and continue execution.\\

\texttt{cofail} & Save the current state information in the current
co-expression block, restore state information from the co-expression
block being activated, perform a context switch, and continue
execution with co-expression failure signal set.\\

\texttt{col n}  & Set the current column number to \texttt{n}.\\

\texttt{compl} & \texttt{\textasciitilde $expr$}\\

\texttt{coret} & Save the current state information in the current
co-expression block, restore state information from the co-expression
block being activated, perform a context switch, and continue
execution with co-expression return signal set.\\

\texttt{create} & Allocate a co-expression block and a refresh block. Copy the
current procedure frame marker, argument values, and local identifier
values into the refresh block. Place a procedure frame for the current
procedure on the stack of the new co-expression block.\\

\texttt{cset a} & Push a descriptor for the cset block at icode offset
\texttt{a} onto the stack. Becomes opcode acset on subsequent executions.\\

\texttt{diff}  & \texttt{$expr_1$ -{}- $expr_2$}\\

\texttt{div}   & \texttt{$expr_1$ / $expr_2$}\\

\texttt{dup}   & Push a null descriptor onto the stack and then push a copy of
the descriptor that was previously on top of the stack.\\

\texttt{efail} & If there is a generator frame in the current expression
frame, resume its generator. Otherwise remove the current expression
frame. If the \texttt{ipc} in its marker is nonzero, set \texttt{ipc} to it. If the
failure \texttt{ipc} is zero, repeat \texttt{efail}.\\

\texttt{\color{blue}einit} & end of initial section. In concurrent
 VMs, change the \texttt{init} instruction to a \texttt{goto} here. \\

\texttt{eqv}   & \texttt{$expr_1$ === $expr_2$}\\

\texttt{eret}  & Save the descriptor on the top of the stack. Unwind the C
stack. Remove the current expression frame from the stack and push the
saved descriptor.\\

\texttt{escan} & Dereference the top descriptor on the stack if necessary. Copy
it to the place on the stack prior to the saved values of \texttt{\&subject} and
\texttt{\&pos} (see \texttt{bscan}). Exchange the current values of
\texttt{\texttt{\&subject}} and \&pos with the saved values on the stack. Then
suspend. If resumed, restore the values of \texttt{\&subject} and \texttt{\&pos}
from the stack and fail.\\

\texttt{esusp} & Create a generator frame containing a copy of the portion of
the stack that is needed if the generator is resumed.\\

\texttt{field n} & Replace the record descriptor on the top of the stack by a
descriptor for field \texttt{n} of that record.\\

\texttt{global n} & Push a variable descriptor pointing to global
identifier \texttt{n}. Becomes opcode aglobal on subsequent executions.\\

\texttt{goto n} & Set \texttt{ipc} to offset \texttt{n}.
Becomes opcode agoto on subsequent executions.\\

\texttt{init n} & Change \texttt{init} instruction to \texttt{goto}.
{\color{blue}Concurrent VMs impose mutual exclusion at this point.}\\

\texttt{int n}  & Push a descriptor for the integer \texttt{n}.\\

\texttt{inter}  & \texttt{$expr_1$ ** $expr_2$}\\

\texttt{invoke n} & \texttt{$expr_0$($expr_1$,$expr_2$,\dots,$expr_n$)}\\

\texttt{keywd n} & Push a descriptor for keyword \texttt{n}.\\

\texttt{lconcat} & \texttt{$expr_1$ ||| $expr_2$}\\

\texttt{lexeq}   & \texttt{$expr_1$ == $expr_2$}\\

\texttt{lexge}   & \texttt{$expr_1$ >= $expr_2$}\\

\texttt{lexgt}   & \texttt{$expr_1$ > $expr_2$}\\

\texttt{lexle}   & \texttt{$expr_1$ <= $expr_2$}\\

\texttt{lexlt}   & \texttt{$expr_1$ < $expr_2$}\\

\texttt{lexne}   & \texttt{$expr_1$ \textasciitilde== $expr_2$}\\

\texttt{limit}   & Convert the descriptor on the top of the stack to an
integer. If the conversion cannot be performed or if the result is
negative, terminate execution with an error message. If it is zero,
fail. Otherwise, create an expression frame with a zero failure \texttt{ipc}.\\

\texttt{line n}  & Set the current line number to \texttt{n}.\\

\texttt{llistn } & \texttt{[$expr_1$,$expr_2$,\dots,$expr_n$]}\\

\texttt{localn } & Push a variable descriptor pointing to local identifier \texttt{n}.\\

\texttt{lsusp}   & Decrement the current limitation counter, which is
immediately prior to the current expression frame on the stack. If the
limit counter is nonzero, create a generator frame containing a copy
of the portion of the interpreter stack that is needed if the
generator is resumed. If the limitation counter is zero, unwind the C
stack and remove the current expression frame from the stack.\\

\texttt{mark}    & Create an expression frame whose marker contains the failure
\texttt{ipc} corresponding to the label \texttt{n}, the current \texttt{efp},
\texttt{gfp}, and \texttt{ilevel}.\\


\texttt{mark0}   & Create an expression frame with a zero failure \texttt{ipc}.\\

\texttt{minus}   & \texttt{$expr_1$ - $expr_2$}\\

\texttt{mod}     & \texttt{$expr_1$ \% $expr_2$}\\

\texttt{mult}    & \texttt{$expr_1$ * $expr_2$}\\

\texttt{neg}     & \texttt{{}-$expr$}\\

\texttt{neqv}    & \texttt{$expr_1$ \textasciitilde=== $expr_2$}\\

\texttt{nonnull} & \texttt{{\textbackslash}$expr$}\\

\texttt{\color{blue}noop}    & \textcolor{blue}{Do nothing.
Allows insertion of labels at certain points. Can be optimized away.} \\

\texttt{null}    &\texttt{/$expr$}\\

\texttt{number}  & \texttt{+$expr$}\\

\texttt{numeq}   & \texttt{$expr_1$ = $expr_2$}\\

\texttt{numge}   & \texttt{$expr_1$ >= $expr_2$}\\

\texttt{numgt}   & \texttt{$expr_1$ > $expr_2$}\\

\texttt{numle}   & \texttt{$expr_1$ <= $expr_2$}\\

\texttt{numlt}   & \texttt{$expr_1$ < $expr_2$}\\

\texttt{numne}   & \texttt{$expr_1$ \textasciitilde= $expr_2$}\\

\texttt{pfail}   & If \texttt{\&trace} is nonzero, decrement it and produce a trace
message. Unwind the C stack and remove the current procedure frame
from the stack. Then fail.\\

\texttt{plus}    & \texttt{$expr_1$ + $expr_2$}\\

\texttt{pnull}   & Push a null descriptor.\\

\texttt{pop}     & Pop the top descriptor.\\

\texttt{power}   & \texttt{$expr_1$ \textasciicircum\ $expr_2$}\\

\texttt{pret}    & Dereference the descriptor on the top of the stack, if
necessary, and copy it to the place where the descriptor for the
procedure is. If \texttt{\&trace} is nonzero; decrement it and produce a trace
message. Unwind the C stack and remove the current procedure frame
from the stack.\\

\texttt{psusp}   & Copy the descriptor on the top of the stack to the place
where the descriptor for the procedure is, dereferencing it if
necessary. Produce a trace message and decrement \texttt{\&trace} if it is
nonzero. Create a generator frame containing a copy of the portion of
the stack that is needed if the procedure call is resumed.\\

\texttt{push1}   & Push a descriptor for the integer 1.\\

\texttt{pushn1}  & Push a descriptor for the integer -1.\\

\texttt{quit}    & Exit from the interpreter.\\

\texttt{random}  & \texttt{?$expr$}\\

\texttt{rasgn}   & \texttt{$expr_1$ <- $expr_2$}\\

\uniconop{rcv}\\
\uniconop{rcvbk}\\

\texttt{real n}  & Push a descriptor for the real number block at offset
\texttt{n} onto the stack.  Becomes opcode areal on subsequent executions.\\

\texttt{refresh} & \texttt{\textasciicircum\ $expr$}\\

\texttt{rswap}   & \texttt{$expr_1$ <-> $expr_2$}\\

\texttt{sdup}    & Push a copy of the descriptor on the top of the stack.\\

\texttt{sect}    & \texttt{$expr_1$ [$expr_2$ : $expr_3$ ]}\\

\texttt{size}    & \texttt{*$expr$}\\

\texttt{static n} & Push a variable descriptor pointing to static
identifier at offset \texttt{n}. Becomes opcode astatic on subsequent
executions.\\

\uniconop{snd}\\
\uniconop{sndbk}\\

\texttt{str n, a} & Push a descriptor for the string of length \texttt{n} 
at offset \texttt{a}. Becomes opcode astr on subsequent executions. \\

\texttt{subsc}   & \texttt{$expr_1$[$expr_2$]}\\

\texttt{swap}    & \texttt{$expr_1$ :=: $expr_2$}\\

\texttt{\color{blue}synt}  &  \textcolor{blue}{ Syntax construct (Pseudo instruction
that assists the Unicon debugger).}\\

\texttt{tabmat}  & \texttt{=$expr$}\\

\texttt{toby}    & \texttt{$expr_1$ to $expr_2$ by $expr_3$}\\

\texttt{unions}  & \texttt{$expr_1$ ++ $expr_2$}\\

\texttt{unmark}  & Remove the current expression frame from the stack and
unwind the C stack.\\

\texttt{value}   & \texttt{.$expr$}\\

\texttt{var}     & Pseudo-instruction; declares variables in ucode.\\
\end{xtabular}

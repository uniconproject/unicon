\chapter{ODBC Database Facilities}

\section{Organization}

The implementation of the ODBC interface includes changes to several
files of the Unicon runtime system, as well as the addition of a new
file for the new functions that were added.

\subsection{New files}

    FDB.R: This is the main file of Unicon ODBC. It contains the Unicon ODBC function set implementation. It is written in standard C with RTT extensions.
    RDB.R: contains the C implementation of odbcerror function that is widely called in FDB.R.

\subsection{Modified files}

    RPROTO.H: contains odbcerror function definition.
    OMISC.R: "*" operator implementation for ODBC file type.
    FDEFS.H: ODBC function definitions.
    DATA.R: runerr error code for ODBC file mismatch.
    RSTRUCTS.H: ISQLFile definition (ODBC connection type).
    REXTERNS.H: ISQLEnv extern definition.
    RMACROS.H: Fs\_ODBC file status flag and ODBC error codes.
    SYS.H: VisualC++ ODBC header files inclusion (windows.h and sqlext.h).
    INIT.R: ODBC Environment structure release.
    DEFINE.H: ISQL symbol definition for conditional compilation.
    GRTTIN.H: new ODBC types definitions.
    MAKEFILE.RUN: Runtime system makefile (FDB.R and RDB.R definitions added)
    ICONX.LNK: Link file (XFDB.OBJ and XRDB.OBJ definitions added)

\section{ISQLFile type}

In Unicon an ODBC connection to a database is similar to a file
operation. Internally this is represented by the following C
structure:

\begin{iconcode}
\#ifdef ISQL\>\>\>\>\>\>\>\>\>\>             /* ODBC support */ \\
\>  struct ISQLFile \{\>\>\>\>\>\>\>\>\>     /* SQL file     */ \\
\>\>    SQLHDBC hdbc;\>\>\>\>\>\>\>\>        /* connection handle */ \\
\>\>    SQLHSTMT hstmt;\>\>\>\>\>\>\>\>      /* statement handle  */ \\
\>\>    char *query;\>\>\>\>\>\>\>\>         /* SQL query buffer */ \\
\>\>    long qsize;\>\>\>\>\>\>\>\>          /* SQL query buffer size */ \\
\>\>    char *tablename; \\
\>\>    struct b\_proc *proc;\>\>\>\>\>\>\>\>/* record constructor for current query */ \\
\>\>    int refcount; \\
\>\>    struct ISQLFile *previous, *next; \\
\>\>    \}; \\
\#endif
\end{iconcode}

The field \texttt{hdbc} is used to keep the connection information associated
to a particular ISQLFile file. \texttt{hstmt} is the statement structure that
saves the results or dataset returned by an ODBC operation. The design
of the interface is table oriented, which means that for each table we
open a new connection.

In the future we will consider the possibility to associate a file to
a database. This would let us open a connection for each database and
share the same connection for each table within the same database. In
this way we can open a file and use more than a table.

Actually when open(DSN,"o",user,password) is called Unicon allocates
an ISQLFile object and initializes the structure fields in the
following way:

\begin{itemize}
\item    hdbc is related to the DSN specified in open()
\item    hstmt is related to hdbc
\end{itemize}

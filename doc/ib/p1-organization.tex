\chapter{Organization of the Implementation}
\label{Org-Chapter}

\textsc{Perspective}: Many factors influence the implementation of a
programming language. The properties of the language itself, of
course, are of paramount importance. Beyond this, goals, resources,
and many other factors may affect the nature of an implementation in
significant and subtle ways.

In the case of the implementation of Icon described here, several
unusual factors deserve mention. To begin with, Icon's origins were in
a research project, and its implementation was designed not only to
make the language available for use but also to support further
language development. The language itself was less well defined and
more subject to modification than is usually the case with an
implementation. Therefore, flexibility and ease of modification were
important implementation goals.

Although the implementation was not a commercial enterprise, neither
was it a toy or a system intended only for a few ``friendly users.''
It was designed to be complete, robust, easy to maintain, and
sufficiently efficient to be useful for real applications in its
problem domain.

Experience with earlier implementations of SNOBOL4, SL5, and the
Ratfor implementation of Icon also influenced the implementation that
is described here. They provided a repertoire of proven techniques and
a philosophy of approach to the implementation of a programming
language that has novel features.

The computing environment also played a major role. The implementation
started on a PDP-11/70 running under UNIX. The UNIX environment
(Ritchie and Thompson 1978), with its extensive range of tools for
program development, influenced several aspects of the implementation
in a direct way. C (Kernighan and Ritchie 1978) is the natural
language for writing such an implementation under UNIX, and its use
for the majority of Icon had pervasive effects, which are described
throughout this book. Tools, such as the Yacc parser-generator
(Johnson 1975), influenced the approach to the translation portion of
the implementation.

Since the initial work was done on a PDP-11/70, with a user address
space of only l28K bytes (combined instruction and data spaces), the
size of the implementation was a significant concern. In particular,
while the Ratfor implementation of Icon fit comfortably on computers
with large address spaces, such as the DEC-10, CDC Cyber, and IBM 370,
this implementation was much too large to fit on a PDP-11/70.

\section{The Icon Virtual Machine}

The implementation of Icon is organized around a virtual machine
(Newey, Poole, and Waite 1972; Griswold 1977). Virtual machines,
sometimes called abstract machines, serve as software design tools for
implementations in which the operations of a language do not fit a
particular computer architecture or where portability is a
consideration and the attributes of several real computer
architectures can be abstracted in a single common model. The
expectation for most virtual machine models is that a translation will
be performed to map the virtual machine operations onto a specific
real machine. A virtual machine also provides a basis for developing
an operational definition of a programming language in which details
can be worked out in concrete terms.

During the design and development phases of an implementation, a
virtual machine serves as an idealized model that is free of the
details and idiosyncrasies of any real machine. The virtual machine
can be designed in such a way that treatment of specific,
machine-dependent details can be deferred until it is necessary to
translate the implementation of the virtual machine to a real one.

Icon's virtual machine only goes so far. Unlike the SNOBOL4 virtual
machine (Griswold 1972), it is incomplete and characterizes only the
expression-evaluation mechanism of Icon and computations on Icon
data. It does not, \textit{per se, }include a model for the
organization of memory. There are many aspects of the Icon run-time
system, such as type checking, storage allocation and garbage
collection, that are not represented in the virtual machine. Instead
Icon's virtual machine serves more as a guide and a tool for
organizing the implementation than it does as a rigid structure that
dominates the implementation.


\section{\textbf{Components of the Implementation}}

There are three major components of the virtual machine implementation
of Icon: a translator, a linker, and a run-time system. The translator
and linker are combined to form a single executable program, but they
remain logically independent.

The translator plays the role of a compiler for the Icon virtual
machine. It analyzes source programs and converts them to virtual
machine instructions. The output of the translator is called
\textit{ucode}. Ucode is represented as ASCII which is helpful in
debugging the implementation.

The linker combines one or more ucode files into a single program for
the virtual machine. This allows programs to be written and translated
in a number of modules, and it is particularly useful for giving users
access to pretranslated libraries of Icon procedures. The output of
the linker, called \textit{icode, }is in binary format for compactness
and ease of processing by the virtual machine. Ucode and icode
instructions are essentially the same, differing mainly in their
format.

Translating and linking are done in two phases:

\begin{picture}(400, 60)(10,0)
\put(-5,30){Icon program}
\put(100,19){\framebox(60,30){translator}}
\put(210,30){ucode}
\put(280,19){\framebox(60,30){linker}}
\put(380,30){icode}
\multiput(75,33)(90,0){4}{\vector(1,0){20}}
\end{picture}

These phases can be performed separately. If only the first phase is
performed, the result is ucode, which can be saved and linked at
another time.

The run-time system consists of an interpreter for icode and a library
of support routines to carry out the various operations that may occur
when an Icon program is executed. The interpreter serves,
conceptually, as a software realization of the Icon virtual
machine. It decodes icode instructions and their operands and carries
out the corresponding operations.

It is worth noting that the organization of the Icon system does not
depend in any essential way on the use of an interpreter. In fact, in
the early versions of this implementation, the linker produced
assembly-language code for the target machine. That code then was
assembled and loaded with the run-time library. On the surface, the
generation of machine code for a specific target machine rather than
for a virtual machine corresponds to the conventional compilation
approach. However, this is somewhat of an illusion, since the machine
code consists largely of calls to run-time library routines
corresponding to virtual machine instructions. Execution of machine
code in such an implementation therefore differs only slightly from
interpretation, in which instruction decoding is done in software
rather than in hardware. The difference in speed in the case of Icon
is relatively minor.

An interpreter offers a number of advantages over the generation of
machine code that offset the small loss of efficiency. The main
advantage is that the interpreter gets into execution very quickly,
since it does not require a loading phase to resolve assembly-language
references to library routines. Icode files also are much smaller than
the executable binary files produced by a loader, since the run-time
library does not need to be included in them. Instead, only one
sharable copy of the run-time system needs to be resident in memory
when Icon is executing.


\section{The Translator}

The translator that produces ucode is relatively conventional. It is
written entirely in C and is independent of the architecture of the
target machine on which Icon runs. Ucode is portable from one target
machine to another.

The translator consists of a lexical analyzer, a parser, a code
generator, and a few support routines. The lexical analyzer converts a
source-language program into a stream of tokens that are provided to
the parser as they are needed.  The parser generates abstract syntax
trees on a per-procedure basis. These abstract syntax trees are in
turn processed by the code generator to produce ucode. The parser is
generated automatically by Yacc from a grammatical specification.
Since the translator is relatively conventional and the techniques
that it uses are described in detail elsewhere (Aho, Lam, Sethi, and
Ullman 2006), it is not discussed here.

There is one aspect of lexical analysis that deserves mention. The
body of an Icon procedure consists of a series of expressions that are
separated by semicolons. However, these semicolons usually do not need
to be provided explicitly, as illustrated by examples in Chapter
2. Instead, the lexical analyzer performs semicolon insertion. If a
line of a program ends with a token that is legal for ending an
expression, and if the next line begins with a token that is legal for
beginning an expression, the lexical analyzer generates a semicolon
token between the lines. For example, the two lines

\begin{iconcode}
\>i := j\textit{ }+ 3\\
\>write(i)
\end{iconcode}

\noindent are equivalent to

\begin{iconcode}
\>i := j + 3;\\
\>write(i)
\end{iconcode}

\noindent since an integer literal is legal at the end of an
expression and an identifier is legal at the beginning of an
expression.

If an expression spans two lines, the place to divide it is at a token
that is not legal at the end of a line. For example,

\begin{iconcode}
\>s1 := s2 ||\\
\>\>s3
\end{iconcode}

\noindent is equivalent to

\iconline{
\>s1 := s2 || s3
}

\noindent No semicolon is inserted, since
\texttt{{\textbar}{\textbar}} is not legal at the end of an
expression.

\section{The Linker}

The linker reads ucode files and writes icode files. An icode file
consists of an executable header that loads the run-time system,
descriptive information about the file, operation codes and operands,
and data specific to the program. The linker, like the translator, is
written entirely in C. While conversion of ucode to icode is largely a
matter of reformatting, the linker performs two other functions.

\subsection{Scope Resolution}

The scope of an undeclared identifier in a procedure depends on global
declarations (explicit or implicit) in the program in which the
procedure occurs. Since the translator in general operates on only one
module of a program, it cannot resolve the scope of undeclared
identifiers, because not all global scope information is contained in
any one module. The linker, on the other hand, processes all the
modules of a program, and hence it has the task of resolving the scope
of undeclared identifiers.

An identifier may be global for several reasons:

\liststyleLii
\begin{itemize}
\item 
\ As the result of an explicit global declaration.
\item 
\ As the name in a record declaration.
\item 
\ As the name in a procedure declaration.
\item 
\ As the name of a built-in function.
\end{itemize}

If an identifier with no local declaration falls into one of these
categories, it is global. Otherwise it is local.

\subsection{Construction of Run-Time Structures}

A number of aspects of a source-language Icon program are represented
at run time by various data structures. These structures are described
in detail in subsequent chapters. They include procedure blocks,
strings, and blocks for cset and real literals that appear in the
program.


This data is represented in ucode in a machine-independent
fashion. The linker converts this information into binary images that
are dependent on the architecture of the target computer.

\section{The Run-Time System}

Most of the interesting aspects of the implementation of Icon reside
in its run-time system. This run-time system is written mostly in C,
although there are a few lines of assembly-language code for checking
for arithmetic overflow and for co-expressions. The C portion is
mostly machine-independent and portable, although some
machine-specific code is needed for some idiosyncratic computer
architectures and to interface some operating-system environments.

There are two main reasons for concentrating the implementation in the
run-time system:

\liststyleLiii
\begin{itemize}

\item Some features of Icon do not lend themselves to translation
directly into executable code for the target machine, since there is
no direct image for them in the target-machine architecture. The
target machine code necessary to carry out these operations therefore
is too large to place in line; instead, it is placed in library
routines that are called from in-line code. Such features range from
operations on structures to string scanning.

\item Operations that cannot be determined at translation time must be
done at run time. Such operations range from type checking to storage
allocation and garbage collection.

\end{itemize}

The run-time system is logically divided into four main parts:
initialization and termination routines, the interpreter, library
routines called by the interpreter, and support routines called by
library routines.


\textbf{Initialization and Termination Routines.}\ \ The
initialization routine sets up regions in which objects created at run
time are allocated. It also initializes some structures that are used
during program execution. Once these tasks are completed, control is
transferred to the Icon interpreter.

When a program terminates, either normally or because of an error,
termination routines flush output buffers and return control to the
operating system.

\textbf{The Interpreter.} The interpreter analyzes icode instructions
and their operands and performs corresponding operations. The
interpreter is relatively simple, since most complex operations are
performed by library routines. The interpreter itself is described in
Chapter 8.

\textbf{Library Routines. }Library routines are divided into three
categories, depending on the way they are called by the interpreter:
routines for Icon operators, routines for Icon built-in functions, and
routines for complicated virtual machine instructions.

The meanings of operators are known to the translator and linker, and
hence they can be called directly. On the other hand, the meanings of
functions cannot be determined until they are executed, and hence they
are called indirectly.

\textbf{Support Routines. }Support routines include storage allocation
and garbage collection, as well as type checking and conversion. Such
routines typically are called by library routines, although some are
called by other support routines.

\textsc{Retrospective}: Superficially, the implementation of Icon
appears to be conventional. An Icon program is translated and linked
to produce an executable binary file. The translator and linker
\textit{are }conventional, except that they generate code and data
structures for a virtual machine instead of for a specific computer.

The run-time system dominates the implementation and plays a much
larger role than is played by run-time systems in conventional
implementations. This run-time system is the focus of the remainder of
this book.

\bigskip

\noindent\textbf{EXERCISES}

\textbf{\ref*{Org-Chapter}.1}
Explain why there is only a comparatively small difference
in execution times between a version of Icon that generates
assembly-language code and one that generates virtual machine code
that is interpreted.

\textbf{\ref*{Org-Chapter}.2}
List all the tokens in the Icon grammar that are legal as the
beginning of an expression and as the end of an expression. Are there
any tokens that are legal as both? As neither?

\textbf{\ref*{Org-Chapter}.3}
Is a semicolon inserted by the lexical analyzer between the
following two program lines?

\begin{iconcode}
\>s1 := s2\\
\>|| s3
\end{iconcode}

\textbf{\ref*{Org-Chapter}.4}
Is it possible for semicolon insertion to introduce
syntactic errors into a program that would be syntactically correct
without semicolon insertion?

% Question 3.5 removed because the translater and linker have been merged.

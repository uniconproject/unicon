\chapter{Differences between Icon and Unicon}

\label{Unicon-Icon}
This appendix summarizes the known differences between Arizona Icon and
Unicon. Since the language has myriad additions covered by the whole book,
the emphasis of this page is on {\em incompatibilities\/} that might
require changes to existing Icon programs.

\section{Extensions to Functions and Operators}

Unicon broadens the meaning of certain pre-existing functions where it
is consistent and unambiguous to do so. These extensions revolve
primarily around the list type. For example,
\index{insert()}\texttt{insert()} allows insertion into the middle of a
list, \texttt{reverse()} reverses a list, and so forth.

\section{Objects}

Unicon supports the concepts of classes and packages with declaration
syntax. This affects scope and visibility of variable names at compile
time. At runtime, objects behave similar to records in most respects.
These additions include reserved words that are no longer valid variable
names, such as \texttt{class}, \texttt{package}, and \texttt{import}.

\section{System Interface}

Unicon's system interface presumes the availability of
hierarchical directory structure, communication between programs using
standard Internet protocols, and other widely available facilities not
present in Arizona Icon.

Unicon's graphics include extensive 3D facilities.  The 2D facilities
are extended with additional image file formats.

\section{Database Facilities}

Unicon supports GDBM and \index{SQL}SQL \index{database}databases with
built-in functions and operators or the experimental SQLite plugin.
The programmer manipulates data in
terms of persistent table and record abstractions. SQL database support
may not be present on platforms that do not provide \index{ODBC}ODBC
open database connectivity drivers or the SQLite plugin.

\section{Multiple Programs and Execution Monitoring Support}

Unicon virtual machine interpreters by default support the loading of
multiple programs so that various debugging and profiling tools can be
applied to them without recompilation. The execution monitoring
facilities are described in "Program Monitoring and
Visualization: An Exploratory Approach", by Clinton
Jeffery. Unicon optimizing compilers may
omit or substitute for these facilities.

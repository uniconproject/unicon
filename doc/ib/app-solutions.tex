\chapter{Solutions to Selected Exercises}

\begin{itemize}
\item[\ref*{ILO-Chapter}.2]
The expression
\iconline{write("hello" \textbar\ "howdy")}
just writes \texttt{hello}; there is no context to cause the second argument of the
alternation to be produced.

\goodbreak\item[\ref*{ILO-Chapter}.3]
The expression
\iconline{\textbar\ (1 to 3) > 10}
is an evaluation "black hole." The left argument produces 1, 2, 3, 1, 2, 3,
1, 2, 3, ... indefinitely, since the comparison always fails.

\goodbreak\item[\ref*{Org-Chapter}.1]
The difference in execution time between interpretation and the execution
of compiled code is comparatively small, since most of the time in execution is spent
in subroutines that are called in either case.

\goodbreak\item[\ref*{Org-Chapter}.3]
The lexical analyzer inserts a semicolon between the lines
\begin{iconcode}
s1 := s2\\
\textbar\textbar\ s3
\end{iconcode}
because an identifier is legal at the end of an expression, and the unary
operator for repeated alternation is legal at the beginning of an expression. 
Thus, the second line consists of two applications of repeated alternation to \texttt{s3}.

\goodbreak\item[\ref*{VV-Chapter}.6]
Type checking is a byproduct of conversion. If conversion were not
automatic, type checking would be simple, but the conversion routines
would still be needed for explicit type conversion. Without implicit type
conversion, the programmer would have to perform explicit conversions,
as in
\iconline{while sum +:= numeric(read())}
However, since both read and numeric can fail, a more complicated construction
would be advisable:
\begin{iconcode}
while x := read() do\\
\>sum +:= numeric(x) \textbar\ stop("nonnumeric data")
\end{iconcode}
This illustrates why the failure of an implicit conversion is an error.

\goodbreak\item[\ref*{VV-Chapter}.7]
A test is necessary to distinguish between qualifiers and other descriptors,
regardless of which has the flags. However, if qualifiers had flags, it
would be necessary to mask them out in order to extract their lengths.
Since masking is needed anyway to extract the type codes of other
descriptors, it is more efficient to relegate the flags to them.

\goodbreak\item[\ref*{VV-Chapter}.9]
 The basic approach to designing one-word descriptors is to have them
point to intermediate data that otherwise would be contained in two-word
descriptors. In the simplest form, one-word descriptors would point to
two-word blocks corresponding to two-word descriptors (Hanson 1980).

\goodbreak\item[\ref*{SC-Chapter}.1]
 If Icon is implemented on a computer whose collating sequence is different 
from ASCII, such as EBCDIC, it still uses the ASCII collating
sequence for comparison and sorting. Consequently, these operations may
produce results that are different from ones that are expected in an
EBCDIC environment. If the native character set of a computer is smaller
than 256, there may be problems implementing Icon, since the size of a C
char may be less than expected in some routines.

\goodbreak\item[\ref*{SC-Chapter}.4]
Even if a newly created string exists in string space, the code to find it
would be complex and the time required probably would outweigh any
advantages of avoiding duplicate allocation. Furthermore, the new string
must be constructed somewhere first in order to know what to look for.
This construction often is most conveniently performed in allocated string
space, so most of the work is done already.

\goodbreak\item[\ref*{SC-Chapter}.7]
If the context in which a subscripting expression is used can be determined
by the translator, different virtual machine instructions could be
generated for the different contexts. If the context is dereferencing, a virtual
machine instruction for producing a substring could be used to avoid
the construction of a substring trapped variable and the allocation of
space that might have to be reclaimed during garbage collection. Similarly,
if the context is assignment, virtual machine instructions for the
appropriate concatenation could be produced.

\goodbreak\item[\ref*{SC-Chapter}.10]
The following procedure illustrates the usefulness of polymorphic subscripting:
\begin{iconcode}
procedure shuffle(x)\\
\>every !x := ?x\\
\>return x\\
end
\end{iconcode}
This procedure can shuffle strings, lists, and records. Note that subscripting
is implicit in the element-generation and random-selection operations.
The same principle applies to explicit subscripting, however.

\goodbreak\item[\ref*{List-Chapter}.2]
An empty list consists of a list-header block and one list-element block.
For convenience in diagramming, the list-element blocks in Chapter 6
have space for only four elements, but the normal number is eight. Therefore,
on a computer with 32-bit words, a list-element block occupies 100
bytes, while a list-header block occupies 24 bytes. Thus, on a computer
with 32-bit words, an empty list occupies 124 bytes in all. On a computer
with 16-bit words, the empty list is half this big. This may seem like a lot
of space for "nothing," but an empty list usually is created so that elements
can be added to it, in which case the overhead becomes progressively
less significant.

\goodbreak\item[\ref*{List-Chapter}.4]
Just because a list element is removed from a list by a stack or queue
operation does not mean that the element is inaccessible. Consider the
expression
\iconline{process(a[1], pop(a))}
Since the arguments are not dereferenced until both of them are
evaluated, it might prove surprising if the value of a[1] were changed by
the pop. Although such situations are unlikely in practice, it is important
that list accesses behave coherently and consistently. One way to view
the handling of this situation is that the pop does not remove an element
but only makes it unavailable for {\em subsequent} access.

\goodbreak\item[\ref*{List-Chapter}.5]
As indicated by the preceding solution, references to list elements that are
removed by stack or queue operations may remain. It is easy to construct
examples in which there are such references to unlinked list-element
blocks.

\goodbreak\item[\ref*{ST-Chapter}.3]
An empty set consists of just a set-header block. There are two words for
the title and size. A computer with 16-bit words has thirteen 4-byte slots,
while a computer with 32-bit words has thirty-seven 8-byte slots. Thus,
on a computer with 16-bit words, an empty set occupies 56 bytes, while
on a computer with 32-bit words, an empty set occupies 304 bytes. Each
member added to a set occupies 12 bytes on a computer with 16-bit words
and 24 bytes on a computer with 32-bit words.

\goodbreak\item[\ref*{ST-Chapter}.8]
It is necessary to access a table to determine whether an element is
already in it. Since the table contains the default value, it is readily available
when it is needed. Because the default value may not be needed, it is
not worth the trouble always to put it in table-element trapped-variable
blocks.

\goodbreak\item[\ref*{ST-Chapter}.12]
The default value of a table is constant and could be used for computing
its hash number.

\goodbreak\item[\ref*{ST-Chapter}.13]
If hashing is based on an attribute of a value that can change, two hash
computations on that value may produce different results. If this happens,
the value appears to be different in the two cases. Thus, the same value
might be inserted in a set twice, an element that is in a table might not be
found, and so on.

\goodbreak\item[\ref*{IX-Chapter}.1]
The \texttt{str} instruction pushes a descriptor on the interpreter stack. 
The d-word of this descriptor is the string length, and the v-word is a pointer
to its location. Since the interpreter stack grows toward increasing memory
addresses, it is natural and convenient to push the v-word first and to have its
value as the first operand of the str instruction.

\goodbreak\item[\ref*{IX-Chapter}.3]
The two expressions
\iconline{a[?i] := a[?i] + 1}
and
\iconline{a[?i] +:= 1}
generally produce different results.

\goodbreak\item[\ref*{IX-Chapter}.4]
A new operator requires new syntax, which must be handled in the translator.
Since each operator has its own virtual machine instruction, a new
instruction must be added for the new operator. The translator, linker, and
run-time system all must handle this new virtual machine instruction. In
the run-time system, the code for this instruction must be added to the
main switch statement in the interpreter, with an appropriate call to a C
routine to implement the operation.

\goodbreak\item[\ref*{EV-Chapter}.2]
Suspension of functions and operations (but not procedures) increases the
interpreter level. This occurs both in nested generators and generators in
mutual evaluation. Thus, if $expr_i$ is a function or operator that suspends,
the interpreter level is increased by its evaluation in both
\iconline{$expr_2$($expr_1$)}
and
\iconline{$expr_1$ \& $expr_2$}

\goodbreak\item[\ref*{EV-Chapter}.5]
Exercise \ref*{ILO-Chapter}.3 contains an example of a "black hole" as a result of
repeated alternation.

\goodbreak\item[\ref*{EV-Chapter}.7]
If the result of $expr_1$ in
\iconline{every $expr_1$ do $expr_2$}
is not popped, the interpreter stack builds up for every value produced by
$expr_1$. While these descriptors are removed when the loop terminates,
interpreter stack overflow would result in expressions like
\iconline{every 1 to 100000 do $expr_2$}

\goodbreak\item[\ref*{EV-Chapter}.8]
The expressions
\iconline{every $expr_1$}
and
\iconline{$expr_1$ \& \&fail}
are equivalent unless $expr_1$ contains a \texttt{break} or \texttt{next} expression.

\goodbreak\item[\ref*{FPC-Chapter}.1]
If a procedure or function call contains an extra argument that fails, the
call is not performed (more accurately, the previous argument is resumed).

\goodbreak\item[\ref*{FPC-Chapter}.3]
If a variable whose value is on the interpreter stack is returned from a procedure,
the variable must be dereferenced, since the value on the stack
may be overwritten. Thus, arguments and dynamic local identifiers are
dereferenced on procedure return. Static local identifiers are not dereferenced,
since they are not on the stack. There is a good argument why
static local identifiers should be dereferenced, since local identifiers are
expected to be accessible only within the procedures in which they are
declared. Suspension also dereferences variables whose values are on the
stack. These values will not be overwritten, however, since the portion of
the stack containing these values is not removed until the surrounding
expression frame is removed, at which time the variables in it are no
longer accessible. The rationale for dereferencing on suspension is that
there should be no difference, at the source-language call, between a procedure
that returns and a procedure that suspends. Furthermore, if dereferencing
on suspension were not done, the arguments and local
identifiers of the suspending procedure could be modified by the procedure
that calls it. This would violate the semantic concept of "local,"
although, as mentioned previously, static local identifiers are not dereferenced.
This may be viewed as a bug.

\goodbreak\item[\ref*{FPC-Chapter}.4]
The interpreter stack grows upward, regardless of the architecture of the
computer on which Icon is implemented. On a computer with an
upward-growing C stack, the C stack base is in the middle of the co-expression
block, instead of at its end. There is an advantage, in co-expressions,
to having a downward-growing C stack: the interpreter and
C stacks grow toward each other, sharing space, and run out of space only
when they collide. For an upward-growing C stack, either the interpreter
stack or the C stack may run out of space, even if there is unused space in
the other.

\goodbreak\item[\ref*{SM-Chapter}.1]
Having the type code in the d-word of a descriptor allows its type to be
determined without a level of indirection to the block title through its v-word.

\goodbreak\item[\ref*{SM-Chapter}.2]
{\em Answer removed: the question no longer applies.}
%% The keyword trapped-variable block for \texttt{\&subject} is statically allocated in
%% the run-time system, so any scanning operation or assignment to \texttt{\&subject}
%% changes the contents of a block in the statically allocated portion of the
%% run-time system.

\goodbreak\item[\ref*{SM-Chapter}.3]
The values of global and static identifiers are in the icode region, so any
assignment to one of them changes data in the icode region.

\goodbreak\item[\ref*{SM-Chapter}.4]
Global and static identifiers could be in a single array in the icode region.
It is simply more convenient to handle them separately in the translator
and linker.

\goodbreak\item[\ref*{SM-Chapter}.5]
The names of global identifiers are needed for the display function and for
the string invocation of functions as described in Exercise \ref*{FPC-Chapter}.2.

\goodbreak\item[\ref*{SM-Chapter}.6]
The names of static identifiers are in their procedure blocks.

\goodbreak\item[\ref*{SM-Chapter}.17]
The maximum type code is a small integer. Therefore, it is only necessary
that the allocated block region be at an address larger than this value. This
happens automatically because of the way the code for the run-time system
is arranged, although it would be trivially easy to force.

\goodbreak\item[\ref*{SM-Chapter}.19]
If there were more than one pointer on \texttt{quallist} to the same qualifier, the
v-word of that qualifier would be relocated more than once, with an
erroneous result. Such an error would show up when the qualifier was
used in subsequent program execution after garbage collection.

\goodbreak\item[\ref*{SM-Chapter}.21]
{\em Answer removed: the question no longer applies.}
%% The keyword trapped-variable block for \texttt{\&subject} may contain a pointer to
%% the allocated string region. This pointer is processed because the keyword
%% trapped-variable block for \texttt{\&subject} is in the basis. Other such cases
%% could be handled the same way.

\goodbreak\item[\ref*{SM-Chapter}.26]
Recursion in markblock occurs when a block contains a pointer to another
block. Thus, pointer chains cause recursion of corresponding depth. The
following program builds a linked list whose length is the number of lines
in the input file:
\begin{iconcode}
record lines(value, nextl)\\
\\
procedure main()\\
\>head := lines()\\
\>current := head\\
\>while current.value := read() do \{\\
\>\>current.nextl := lines()\\
\>\>current := current.nextl\\
\>\}\\
\>\>\vdots\\
end
\end{iconcode}
In the case of a garbage collection, the depth of recursion in \texttt{markblock}
is the length of the linked list.

\goodbreak\item[\ref*{SM-Chapter}.31]
An excessively large predictive need request may trigger a garbage collection
unnecessarily or possibly cause error termination if there is not
enough memory, although this is unlikely unless the request is enormous.
If the request is satisfied, execution continues normally. If a predictive
need request is less than the amount subsequently allocated and there is
not enough space in the region to satisfy the allocation request, the allocation
routine terminates program execution with a message indicating that
there is an error in the implementation of Icon.

\goodbreak\item[\ref*{SM-Chapter}.32]
As indicated in the solution to Exercise 6.5, an unlinked list-element
block may still be referenced by a variable. As long as this is the case, the
list-element block cannot be reclaimed by garbage collection. In general,
as long as there is a pointer to a block that is reachable from the basis, the
block cannot be reclaimed.

\goodbreak\item[\ref*{SM-Chapter}.33]
A variable could point to the head of the block, with an offset referred to
the location of the corresponding value. The advantage of having a variable
point directly to its value is efficiency during expression evaluation.
Access to variables in blocks is presumably more frequent during expression
evaluation than during garbage collection. There can be at most one
access to such a variable during a garbage collection, while the variable
may be accessed repeatedly during expression evaluation. Furthermore,
many programs use such variables frequently but never require garbage
collection.

\goodbreak\item[\ref*{SM-Chapter}.38]
In the case of a type, such as real, for which all blocks are the same size, a
separate allocation region can be efficiently managed by a free-list
mechanism. No relocation is needed in such a region, and since all blocks
are of the same size, there is no problem with fragmentation. Having
many allocation regions complicates the storage-management system, of
course. The implementation of an early version of Icon employed several
regions in which the space for freed blocks was reused (Hanson 1980).

\goodbreak\item[\ref*{SM-Chapter}.39]
It is not possible to determine, in general, if storage is needed only temporarily
or if it must be retained for an arbitrarily long time. For example, in
\iconline{while process(read)}
the retention of storage for input strings depends entirely on what process
does. If it just writes out its argument, as shown in the exercise, storage
for the strings produced by read would not have to be allocated. If, on
the other hand, it pushes its argument on a list, the strings must be allocated.
Since allocation is fast and garbage collection has little work to do
for space that is not accessible, storage throughput has a comparatively
small impact on performance.

\goodbreak\item[\ref*{RT-Chapter}.2]
On computers with 16-bit words, tests for the two types of integers must
be made in all situations where type checking and conversion are performed.
Furthermore, all computations that may produce a source-language
integer that requires more than 16 bits must provide for the
creation of long-integer blocks. Since integers occur in many contexts, a
substantial amount of code is required to handle the two types of integers.

\goodbreak\item[\ref*{RT-Chapter}.4]
The value of \texttt{MaxCvtLen} must be sufficient to handle a string whose
length is the size of the character set. Thus, if Icon had 512 different characters,
\texttt{MaxCvtLen} would have to be 512. Even if the number of different
characters were small, it is unlikely that some other type of conversion,
such as real-to-string, would dominate.

\goodbreak\item[\ref*{RT-Chapter}.14]
{\em Answer removed: the question no longer applies.}
%% In \texttt{write(x1,x2,\ \dots,\ xn)}, the arguments are converted to strings,
%% if necessary, for the purposes of output. With the exception of \texttt{xn},
%% such strings resulting from conversion do not have to be placed in allocated
%% storage, since they are used only transiently. However, the value returned by
%% \texttt{write} is the string value of \texttt{xn}, so if it is not a string, its
%% converted value must be placed in allocated string storage.

\end{itemize}

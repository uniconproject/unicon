\chapter{Experimental Features}

The designers of Unicon have taken a very conservative approach when adding to
the language and when changing existing features. With the small number of
exceptions that have been previously noted on page \pageref{Unicon-Icon}, an
Icon program that runs on the final version of Icon (version 9.5, first released
in 1996) will run on the current Unicon system {\em and give the same results\/}
a quarter of a century later. The conservative approach is continued when
dealing with additions to Unicon; breaking existing Unicon programs by making an
incompatible change to the language is, in most circumstances, considered to be
a very bad thing to do.

Most of the development of Unicon starting from its progenitor has already been
discussed but there are some more experimental features that are waiting in the
wings. Some of them may never see the light of day in their present form -- or,
perhaps, in any form -- so the most cautious approach is not to rely on any of
them until they make their way from this appendix into the definition of the
language in Appendix A.

Without exception, the experimental features are not enabled by default in a
release build of Unicon -- they can only be accessed by making the appropriate
pre-processor definitions (or, in some cases, by specifying additional arguments
to \texttt{configure}) and rebuilding the system from the source code. Some
features that are now part of the language -- for example, the array extension
to lists that makes them faster in many cases -- are still guarded by
pre-processor definitions, showing their pedigree as experimental additions, but
are now enabled by default.

%% Candidates for inclusion
%%
%% Plugins
%% User defined operators
%% UTF-8
%% Extensions to &random

\section{User defined operators}
This feature extends the syntax of classes to allow the built-in operator
symbols to be redefined when their operands are objects. It may be enabled by
using the \texttt{-{}-enable-ovld} option to \texttt{configure} before
rebuilding the Unicon system.

\section{Extensions to \texttt{\&random}}
This feature allows the programmer to choose from a portfolio of different
random number generators (in addition to the one provided by Icon). It is also
possible to implement other generators and use them without rebuilding Unicon.
More than one generator may be in use at the same time.
It may be enabled by defining the C preprocessor symbol \texttt{RngLibrary}
before rebuilding the Unicon system.

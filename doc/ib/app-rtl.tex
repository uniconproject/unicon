\chapter{The RTL Run-Time Language }

This appendix contains a description of the language used to implement
the run-time operations of the Icon compiler system. Chapter 5
provides a description of the design goals of the implementation
language and an introduction to it.  Some of the design decisions for
the language were motivated by optimizations planned for the future,
such as constant folding of csets. The use of these features is
presented as if the optimizations were implemented; this insures that
the optimizations will be supported by the run-time system when they
are implemented. This appendix is adapted from the reference manual
for the language [Walker94].

The translator for the implementation language is the program rtt. An
rtt input file may contain operation definitions written in the
implementation language, along with C definitions and
declarations. Rtt has a built-in C preprocessor based on the ANSI C
Standard, but with extensions to support multi-line macros with
embedded preprocessor directives [.ipd65.]. Rtt prepends a standard
include file, grttin.h, on the front of every implementation language
file it translates.

The first part of this appendix describes the operation definitions. C
language documentation should be consulted for ordinary C grammar. The
extensions to ordinary C grammar are described in the latter part of
the appendix.

The grammar for the implementation language is presented in extended
BNF notation. Terminal symbols are set in Helvetica. Non-terminals and
meta-symbols are set in \textit{Times-Italic}. In addition to the
usual meta-symbols, \textit{::=} for
{\textasciigrave}{\textasciigrave}is defined as'{}' and
\textit{{\textbar}} for
{\textasciigrave}{\textasciigrave}alternatives'{}', brackets around a
sequence of symbols indicates that the sequence is optional, braces
around a sequence of symbols followed by an asterisk indicates that
the sequence may be repeated zero or more times, and braces followed
by a plus indicates that the enclosed sequence may be repeated one or
more times.

\section{Operation Documentation }

An operation definition can be preceded by an optional description in
the form of a C string literal.

{\ttfamily\mdseries
\ \ \ \ \ \ \ \ \textit{documented-definition ::= [ C-string-literal ] operation-definition}}


The use of a C string allows an implementation file to be run through
the C preprocessor without altering the description. The preprocessor
concatenates adjacent string literals, allowing a multi-line
description to be written using multiple strings. Alternatively, a
multi-line description can be written using {\textasciigrave}' for
line continuation. This description is stored in the operation data
base where it can be extracted by documentation generation
programs. These documentation generators produce formatted
documentation for Icon programmers and for C programmers maintaining
the Icon implementation. The documentation generators are responsible
for inserting newline characters at reasonable points when printing
the description.

\section{Types of Operations }

Rtt can be used to define the built-in functions, operators, and
keywords of the Icon language. (Note that there are some Icon
constructs that fall outside this implementation specification
system. These include control structures such as string scanning and
limitation, along with record constructors and field references.)

{\ttfamily\mdseries
\ \ \ \textit{operation-definition} ::=}

{\ttfamily\mdseries
\ \ \ \ \ \ function result-seq identifier ( [ param-list ] ) [ declare ] actions end {\textbar}}

{\ttfamily\mdseries
\ \ \ \ \ \ operator result-seq op identifier ( [ param-list ] ) [ declare ] actions end {\textbar}}

{\ttfamily\mdseries
\ \ \ \ \ \ keyword result-seq identifier actions end {\textbar}}

{\ttfamily\mdseries
\ \ \ \ \ \ keyword result-seq identifier const key-const end}

{\ttfamily\mdseries
\ \ \ result-seq ::= \{ length , length [ + ] \} {\textbar}}

{\ttfamily\mdseries
\ \ \ \ \ \ \ \ \ \ \ \ \ \ \ \ \ \ \{ length [ + ] \} {\textbar}}

{\ttfamily\mdseries
\ \ \ \ \ \ \ \ \ \ \ \ \ \ \ \ \ \ \{ \}}

{\ttfamily\mdseries
\ \ \ length ::= integer {\textbar} *}


\textit{result-seq} indicates the minimum and maximum length of the
result sequence of an operation (the operation is treated as if it is
used in a context where it produces all of its results). For example,
addition always produces one result so its \textit{result-seq} is \{1,
1\}. If the minimum and maximum are the same, only one number need be
given, so the \textit{result-seq} for addition can be coded as
\{1\}. A conditional operation can produce either no results (that is,
it can fail) or it can produce one result, so its \textit{result-seq}
is \{0, 1\}. A length of $\ast$ indicates unbounded, so the
\textit{result-seq} of ! is indicated by \{0, $\ast$\}. An $\ast$ in the lower
bound means the same thing as 0, so \{0, $\ast$\} can be written as \{$\ast$,$\ast$\},
which simplifies to \{$\ast$\}. A \textit{result-seq} of \{\} indicates no
result sequence.  This is not the same as a zero-length result
sequence, \{0\}; an operation with no result sequence does not even
fail.  exit is an example of such an operation.

A + following the length(s) in a \textit{result-seq} indicates that
the operation can be resumed to perform some side effect after
producing its last result. All existing examples of such operations
produce at most one result, performing a side effect in the
process. The side effect on resumption is simply an undoing of the
original side effect. An example of this is tab, which changes \&pos
as the side effect.

For functions and keywords, \textit{identifier} is the name by which
the operation is known within the Icon language (for keywords,
\textit{identifier} does not include the \&). New functions and
keywords can be added to the language by simply translating
implementations for them. For operations, \textit{op} is (usually) the
symbol by which the operation is known within the Icon language and
\textit{identifier} is a descriptive name. It is possible to have more
than one operation with the same \textit{op} as long as they have
different identifiers and take a different number of operands.  In
addition to translating the implementation for an operator, adding a
new operator requires updating iconc's lexical analyzer and parser to
know about the symbol (in reality, an \textit{operator} definition may
be used for operations with non-operator syntax, in which case any
syntax may be used; iconc's code generator identifies the operation by
the type of node put in the parse tree by a parser action). In all
cases, the \textit{identifier} is used to construct the name(s) of the
C function(s) which implement the operation.


A \textit{param-list} is a comma separated list of parameter
declarations. Some operations, such as the write function, take a
variable number of arguments. This is indicated by appending a pair of
brackets enclosing an identifier to the last parameter
declaration. This last parameter is then an array containing the
\textit{tail} of the argument list, that is, those arguments not taken
up by the preceding parameters. The identifier in brackets represents
the length of the tail and has a type of C integer.

{\ttfamily\mdseries
\ \ \ \textit{param-list ::= param \{ , param \} [ [ identifier ] ]}}


Most operations need their arguments dereferenced. However, some
operations, such as assignment, need undereferenced arguments and a
few need both dereferenced and undereferenced versions of an
argument. There are forms of parameter declarations to match each of
these needs.

{\ttfamily\mdseries
\ \ \ \textit{param ::= identifier {\textbar}}}

{\ttfamily\mdseries
\ \ \ \ \ \ \ \ \ \ \ \ \ \ \ \ \textit{underef identifier {\textbar}}}

{\ttfamily\mdseries
\ \ \ \ \ \ \ \ \ \ \ \ \ \ \ \ \textit{underef identifier -{\textgreater} identifier}}


A simple identifier indicates a dereferenced parameter. underef
indicates an undereferenced parameter. In the third form of parameter
declaration, the first identifier represents the undeferenced form of
the argument and the second identifier represents the dereferenced
form. This third form of declaration may not be used with the variable
part of an argument list. These identifiers are of type
\textit{descriptor}. Descriptors are implemented as C structs. See
Chapter 4 for a detailed explanation of descriptors. Examples of
operation headers:

{\ttfamily\mdseries
\ \ \ detab(s,i,...) - replace tabs with spaces, with stops at columns indicated.}

{\ttfamily\mdseries
\ \ \ function\{1\} detab(s, i[n])}

{\ttfamily\mdseries
\ \ \ \ \ \ \textit{actions}}

{\ttfamily\mdseries
\ \ \ end}


\bigskip

{\ttfamily\mdseries
\ \ \ x {\textless}-{\textgreater} y -swap values of x and y.}

{\ttfamily\mdseries
\ \ \ Reverses swap if resumed.}

{\ttfamily\mdseries
\ \ \ operator\{0,1+\} {\textless}-{\textgreater} rswap(underef x -{\textgreater} dx, underef y -{\textgreater} dy)}

{\ttfamily\mdseries
\ \ \ \ \ \ declare}

{\ttfamily\mdseries
\ \ \ \ \ \ actions}

{\ttfamily\mdseries
\ \ \ end}


\bigskip

{\ttfamily\mdseries
\ \ \ \&fail -just fail}

{\ttfamily\mdseries
\ \ \ keyword\{0\} fail}

{\ttfamily\mdseries
\ \ \ \ \ \ actions}

{\ttfamily\mdseries
\ \ \ end}


\section{Declare Clause}

Some operations need C declarations that are common to several
actions. These can be declared within the declare clause.


{\ttfamily\mdseries
\ \ \ \ \ \ \textit{declare ::= declare \{ C declarations \}}}


These may include \textit{tended} declarations, which are explained
below in the section on extensions to C. If a declaration can be made
local to a block of embedded C code, it is usually better to put it
there than in a declare clause. This is explained below in the
discussion of the body action.

{\sffamily
Constant Keywords}

Any keyword can be implemented using general
\textit{actions}. However, for constant keywords, iconc can sometimes
produce more efficient code if it treats the keyword as a literal
constant. Therefore, a special declaration is available for declaring
keywords that can be represented as Icon literals. The constant is
introduced with the word const and can be one of four literal types.

{\ttfamily\mdseries
\ \ \ \ \ \ \textit{key-const ::= string-literal {\textbar} cset-literal {\textbar} integer-literal {\textbar}
real-literal}}


When using this mechanism, it is important to be aware of the fact
that rtt tokenizes these literals as C literals, not as Icon
literals. The contents of string literals and character literals (used
to represent cset literals) are not interpreted by rtt except for
certain situations in string concatenation (see [.ipd65.]). They are
simply stored, as is, in the data base. This means that literals with
escape sequences can be used even when C and Icon would give them
different interpretations. However, C does not recognize control
escapes, so '\^{}'{}', which is a valid Icon literal, will result in
an error message from rtt, because the second quote ends the literal,
leaving the third quote dangling.  Only decimal integer literals are
allowed.


\section{Actions }

All operations other than constant keywords are implemented with
general \textit{actions}.

Actions fall into four categories: type checking and conversions,
detail code expressed in extended C, abstract type computations, and
error reporting.

{\ttfamily\mdseries
\ \ \ \textit{actions} ::= \{ action \}*}

{\ttfamily\mdseries
\ \ \ action ::= checking-conversions {\textbar}}

{\ttfamily\mdseries
\ \ \ \ \ \ \ \ \ \ \ \ \ \ detail-code {\textbar}}

{\ttfamily\mdseries
\ \ \ \ \ \ \ \ \ \ \ \ \ \ abstract \{ type-computations \} {\textbar}}

{\ttfamily\mdseries
\ \ \ \ \ \ \ \ \ \ \ \ \ \ runerr( msg\_number [ , descriptor ] ) [ ; ]}

{\ttfamily\mdseries
\ \ \ \ \ \ \ \ \ \ \ \ \ \{ actions \}}

\subsection{Type Checking and Conversions}

The type checking and conversions are 

{\ttfamily\mdseries
\ \ \ checking-conversions ::= if type-check then action {\textbar}}

{\ttfamily\mdseries
\ \ \ \ \ \ \ \ \ \ \ \ \ \ \ \ \ \ \ \ \ \ \ \ \ \ \ \ if type-check then action else action {\textbar}}

{\ttfamily\mdseries
\ \ \ \ \ \ \ \ \ \ \ \ \ \ \ \ \ \ \ \ \ \ \ \ \ \ \ \ type\_case descriptor of \{ \{ type-select \}+ \} {\textbar}}

{\ttfamily\mdseries
\ \ \ \ \ \ \ \ \ \ \ \ \ \ \ \ \ \ \ \ \ \ \ \ \ \ \ \ len\_case \textit{identifier} of \{ \{ integer : action \}+
default : action \}}

{\ttfamily\mdseries
\ \ \ type-select ::= \{ type-name : \}+ action {\textbar}}

{\ttfamily\mdseries
\ \ \ \ \ \ \ \ \ \ \ \ \ \ \ \ \ \ \ default : action}


These actions specify run-time operations. These operations could be
performed in C, but specifying them in the implementation language
gives the compiler information it can use to generate better code.

The if actions use the result of a \textit{type-check} expression to
select an action. The type\_case action selects an action based on the
type of a descriptor. If a type\_case action contains a default
clause, it must be last.  \textit{type-select} clauses must be
mutually exclusive in their selection. The len\_case action selects an
action based on the length of the variable part of the argument list
of the operation. The \textit{identifier} in this action must be the
one representing that length.

A \textit{type-check} can succeed or fail. It is either an assertion
of the type of a descriptor, a conversion of the type of a descriptor,
or a logical expression involving \textit{type-check}s. Only limited
forms of logical expressions are supported.

{\ttfamily\mdseries
\ \ \ \textit{type-check} ::= simple-check \{ \&\& simple-check \}* {\textbar}}

{\ttfamily\mdseries
\ \ \ \ \ \ \ \ \ \ \ \ \ \ \ \ \ \ \ \ \ ! simple-check}


\bigskip

{\ttfamily\mdseries
\ \ \ simple-check ::= is: type-name ( descriptor ) {\textbar}}

{\ttfamily\mdseries
\ \ \ \ \ \ \ \ \ \ \ \ \ \ \ \ \ \ \ \ cnv: dest-type ( source [ , destination ] ) {\textbar}}

{\ttfamily\mdseries
\ \ \ \ \ \ \ \ \ \ \ \ \ \ \ \ \ \ \ \ def: dest-type ( source , value [ , destination ] )}


\bigskip

{\ttfamily\mdseries
\ \ \ dest-type ::= cset {\textbar}}

{\ttfamily\mdseries
\ \ \ \ \ \ \ \ \ \ \ \ \ \ \ \ \ integer {\textbar}}

{\ttfamily\mdseries
\ \ \ \ \ \ \ \ \ \ \ \ \ \ \ \ \ real {\textbar}}

{\ttfamily\mdseries
\ \ \ \ \ \ \ \ \ \ \ \ \ \ \ \ \ string {\textbar}}

{\ttfamily\mdseries
\ \ \ \ \ \ \ \ \ \ \ \ \ \ \ \ \ C\_integer {\textbar}}

{\ttfamily\mdseries
\ \ \ \ \ \ \ \ \ \ \ \ \ \ \ \ \ C\_double {\textbar}}

{\ttfamily\mdseries
\ \ \ \ \ \ \ \ \ \ \ \ \ \ \ \ \ C\_string {\textbar}}

{\ttfamily\mdseries
\ \ \ \ \ \ \ \ \ \ \ \ \ \ \ \ \ (exact)integer {\textbar}}

{\ttfamily\mdseries
\ \ \ \ \ \ \ \ \ \ \ \ \ \ \ \ \ (exact)C\_integer {\textbar}}

{\ttfamily\mdseries
\ \ \ \ \ \ \ \ \ \ \ \ \ \ \ \ \ tmp\_string {\textbar}}

{\ttfamily\mdseries
\ \ \ \ \ \ \ \ \ \ \ \ \ \ \ \ \ tmp\_cset}


The is check succeeds if the value of the descriptor is in the type
indicated by \textit{type-name}. Conversions indicated by cnv are the
conversions between the Icon types of cset, integer, real, and
string. Conversions indicated by def are the same conversions with a
default value to be used if the original value is null.

\textit{dest-type} is the type to which to a value is to be converted,
if possible. cset, integer, real, and string constitute a subset of
\textit{icon-type} which is in turn a subset of \textit{type-name}
(see below). C\_integer, C\_string, and C\_double are conversions to
internal C types that are easier to manipulate than descriptors. Each
of these types corresponds to an Icon type. A conversion to an
internal C type succeeds for the same values that a conversion to the
corresponding Icon type succeeds. C\_integer represents the C integer
type used for integer values in the particular Icon implementation
being compiled (typically, a 32-bit integer type). C-double represents
the C double type. C-string represents a pointer to a null-terminated
C character array. However, see below for a discussion of the
destination for conversion to C\_string. (exact) before integer or
C\_integer disallows conversions from reals or strings representing
reals, that is, the conversion fails if the value being converted
represents a real value.

Conversion to tmp\_string is the same as conversion to string (the
result is a descriptor), except that the string is only guaranteed to
exist for the lifetime of the operation (the lifetime of a suspended
operation extends until it can no longer be resumed). Conversion to
tmp\_string is generally less expensive than conversion to string and
is never more expensive, but the resulting string must not be exported
from the operation. tmp\_cset is analogous to tmp\_string.

The source of the conversion is the descriptor whose value is to be
converted. If no destination is specified, the conversion is done
{\textasciigrave}{\textasciigrave}in-place'{}'. However, it may not
actually be possible to do an argument conversion in the argument's
original location, so the argument may be copied to another location
as part of the conversion. Within the \textit{scope} of the
conversion, the parameter name refers to this new location. The scope
of a conversion is usually only important for conversions to C types;
the run-time system translator and the Icon compiler try to keep the
movement of descriptor parameters transparent (see below for more
details). All elements of the variable part of an argument list must
be descriptors. Therefore, when an element is converted to a C type,
an explicit location must be given for the destination.

The destinations for conversions to cset, integer, real, string,
(exact)integer, tmp\_string, and tmp\_cset must be descriptors. The
destinations for conversions to C\_integer, C\_double, and
(exact)C\_integer must be the corresponding C types. However, the
destination for conversion to C\_string must be tended. If the
destination is declared as {\textasciigrave}{\textasciigrave}tended
char '{}', then the dword (string length) of the tended location will
be set, but the operation will not have direct access to it. The
variable will look like a {\textasciigrave}{\textasciigrave}char
'{}'. Because the operation does not have access to the string length,
it is not a good idea to change the pointer once it has been set by
the conversion. If the destination is declared as a descriptor, the
operation has access to both the pointer to the string and the
string's length (which includes the terminating null character). If a
parameter is converted to C\_string and no explicit destination is
given, the parameter will behave like a
{\textasciigrave}{\textasciigrave}tended char '{}' within the scope of
the conversion.

The second argument to the def conversion is the default value. The
default value may be any C expression that evaluates to the correct
type. These types are given in the following chart.

\begin{center}
\tablefirsthead{}
\tablehead{}
\tabletail{}
\tablelasttail{}
\begin{xtabular}{m{0.93025976in}m{0.7538598in}}
 cset:  &
 struct b\_cset \\
 integer:  &
 C\_integer \\
 real:  &
 double \\
 string:  &
 struct descrip \\
 C\_integer:  &
 C\_integer \\
 C\_double:  &
 double \\
 C\_string:  &
 char * \\
 tmp\_string: &
 struct descrip \\
 tmp\_cset:  &
 struct b\_cset \\
 (exact)integer:  &
 C\_integer \\
 (exact)C\_integer:  &
 C\_integer \\
\end{xtabular}
\end{center}

The numeric operators provide good examples of how conversions are used: 

{\ttfamily\mdseries
\ \ \ operator\{1\} / divide(x, y)}

{\ttfamily\mdseries
\ \ \ \ \ \ if cnv:(exact)C\_integer(x) \&\& cnv:(exact)C\_integer(y) then}

{\ttfamily\mdseries
\ \ \ \ \ \ \ \ \ \textit{actions}}

{\ttfamily\mdseries
\ \ \ \ \ \ else \{}

{\ttfamily\mdseries
\ \ \ \ \ \ \ \ \ if !cnv:C\_double(x) then}

{\ttfamily\mdseries
\ \ \ \ \ \ \ \ \ \ \ \ runerr(102, x)}

{\ttfamily\mdseries
\ \ \ \ \ \ \ \ \ if !cnv:C\_double(y) then}

{\ttfamily\mdseries
\ \ \ \ \ \ \ \ \ \ \ \ runerr(102, y)}

{\ttfamily\mdseries
\ \ \ \ \ \ \ \ \ \textit{actions}}

{\ttfamily\mdseries
\ \ \ \ \ \ \ \ \ \}}

{\ttfamily\mdseries
\ \ \ \ end}

Within the code indicated by \textit{actions}, x and y refer to C
values rather than to the Icon descriptors of the unconverted
parameters.

The subject of any type check or type conversion must be an unmodified
parameter. For example, once an in-place conversion has been applied
to a parameter, another conversion may not be applied to the same
parameter. This helps insure that type computations in iconc only
involve the unmodified types of arguments, simplifying those
computations.  This restriction does not apply to type checking and
conversions in C code.

\subsection{Scope of Conversions}

The following discussion is included mostly for completeness. The
scope of conversions sounds complicated, but in practice problems
seldom occur in code that {\textasciigrave}{\textasciigrave}looks
reasonable'{}'. If a problem does occur, the translator catches
it. Normally, the intricacies of scope should be ignored and the
person writing run-time routines should code conversions in a manner
that seems natural.

An {\textquotedbl}in-place{\textquotedbl} conversion of a parameter
can create a scope for the parameter name separate from the one
introduced by the parameter list. This is because conversions to C
types may require the converted value to be placed in a different
location with a different type. The parameter name is then associated
with this new location. The original scope of a parameter starts at
the beginning of the operation's definition. The scope of a conversion
starts at the conversion. A scope extends through all code that may be
executed after the scope's beginning, up to a runerr or a conversion
that hides the previous scope (because the type checking portion of
the implementation language does not contain loops or arbitrary gotos,
scope can easily be determined lexically).

The use of an in-place conversion in the first sub-expression of a
conjunction, \textit{cnv1 \&\& cnv2}, has a potential for causing
problems. In general, there is no way to know whether the first
conversion will effectively be undone when the second conversion
fails. If the first conversion is actually done in-place, the
parameter name refers to the same location in both the success and
failure scope of the conjunction, so the conversion is not undone. If
the conversion is done into a separate location, the failure scope
will refer to the original value, so the conversion will effectively
be undone. Whether the conversion is actually done in-place depends on
the context in which operation is used. However, conversion to
C\_integer and C\_double always preserve the original value, so there
is no potential problem using them as the first argument to a
conjunction, nor is there any problem using a non-conversion test
there.  An example of this uncertainty:

{\ttfamily\mdseries
\ \ \ if cnv:string(s1) \&\& cnv:string(s2) then \{}

{\ttfamily\mdseries
\ \ \ \ \ \ /* s1 and s2 both refer to converted values */}

{\ttfamily\mdseries
\ \ \ \ \ \ \}}

{\ttfamily\mdseries
\ \ \ else \{ /* s2 refers to the original value. s1 may}

{\ttfamily\mdseries
\ \ \ \ \ \ \ \ \ \ \ \ \ refer to either the original or the converted value */}

{\ttfamily\mdseries
\ \ \ \ \ \ \}}


The translator issues a warning if there is a potential problem. 

It is possible for scopes to overlap; this happens because scopes
start within conditional actions. In rare instances, executable code
using the name may appear within this overlapping scope, as in the
following example, which resembles code that might be found in the
definition of a string analysis function such as find.

{\ttfamily\mdseries
\ \ \ if is:null(s) then \{}

{\ttfamily\mdseries
\ \ \ \ \ \ if !def:C\_integer(i, k\_pos) then}

{\ttfamily\mdseries
\ \ \ \ \ \ \ \ \ runerr(101, i)}

{\ttfamily\mdseries
\ \ \ \ \ \ \}}

{\ttfamily\mdseries
\ \ \ else \{}

{\ttfamily\mdseries
\ \ \ \ \ \ if !def:C\_integer(i, 1) then}

{\ttfamily\mdseries
\ \ \ \ \ \ \ \ \ runerr(101, i)}

{\ttfamily\mdseries
\ \ \ \ \ \ \textit{actions}}


Here, \textit{actions} occurs within the scope of both
conversions. Note that \textit{actions} is not in the scope of the
original parameter i. This is because that scope is ended in each
branch of the outer if by the conversions and the runerrs.

If overlap does occur, the translator tries to insure that the same
location is used for the name in each scope. The only situation when
it cannot do this is when the type of the location is different in
each scope, for instance, one is a C\_integer and the other is a
C\_real. If a name is referenced when there is conflicting scope, the
translator issues an error message.

\subsection{Type Names}

The \textit{type-name}s represent types of Icon intermediate values,
including variable references. These are the values that enter and
leave an operation; {\textasciigrave}{\textasciigrave}types'{}'
internal to data structures, such as list element blocks, are handled
completely within the C code.

{\ttfamily\mdseries
\ \ \ \ \ \ \textit{type-name} ::= empty\_type {\textbar}}

{\ttfamily\mdseries
\ \ \ \ \ \ \ \ \ \ \ \ \ \ \ \ \ \ \ \ \ \ \ icon-type {\textbar}}

{\ttfamily\mdseries
\ \ \ \ \ \ \ \ \ \ \ \ \ \ \ \ \ \ \ \ \ \ \ variable-ref}

{\ttfamily\mdseries
\ \ \ \ \ \ icon-type ::= null {\textbar}}

{\ttfamily\mdseries
\ \ \ \ \ \ \ \ \ \ \ \ \ \ \ \ \ \ \ \ string {\textbar}}

{\ttfamily\mdseries
\ \ \ \ \ \ \ \ \ \ \ \ \ \ \ \ \ \ \ \ cset {\textbar}}

{\ttfamily\mdseries
\ \ \ \ \ \ \ \ \ \ \ \ \ \ \ \ \ \ \ \ integer {\textbar}}

{\ttfamily\mdseries
\ \ \ \ \ \ \ \ \ \ \ \ \ \ \ \ \ \ \ \ real {\textbar}}

{\ttfamily\mdseries
\ \ \ \ \ \ \ \ \ \ \ \ \ \ \ \ \ \ \ \ file {\textbar}}

{\ttfamily\mdseries
\ \ \ \ \ \ \ \ \ \ \ \ \ \ \ \ \ \ \ \ list {\textbar}}

{\ttfamily\mdseries
\ \ \ \ \ \ \ \ \ \ \ \ \ \ \ \ \ \ \ \ set {\textbar}}

{\ttfamily\mdseries
\ \ \ \ \ \ \ \ \ \ \ \ \ \ \ \ \ \ \ \ table {\textbar}}

{\ttfamily\mdseries
\ \ \ \ \ \ \ \ \ \ \ \ \ \ \ \ \ \ \ \ record {\textbar}}

{\ttfamily\mdseries
\ \ \ \ \ \ \ \ \ \ \ \ \ \ \ \ \ \ \ \ procedure {\textbar}}

{\ttfamily\mdseries
\ \ \ \ \ \ \ \ \ \ \ \ \ \ \ \ \ \ \ \ co\_expression}

{\ttfamily\mdseries
\ \ \ \ \ \ variable-ref ::= variable {\textbar}}

{\ttfamily\mdseries
\ \ \ \ \ \ \ \ \ \ \ \ \ \ \ \ \ \ \ \ \ \ \ tvsubs {\textbar}}

{\ttfamily\mdseries
\ \ \ \ \ \ \ \ \ \ \ \ \ \ \ \ \ \ \ \ \ \ \ tvtbl {\textbar}}

{\ttfamily\mdseries
\ \ \ \ \ \ \ \ \ \ \ \ \ \ \ \ \ \ \ \ \ \ \ kywdint {\textbar}}

{\ttfamily\mdseries
\ \ \ \ \ \ \ \ \ \ \ \ \ \ \ \ \ \ \ \ \ \ \ kywdpos {\textbar}}

{\ttfamily\mdseries
\ \ \ \ \ \ \ \ \ \ \ \ \ \ \ \ \ \ \ \ \ \ \ kywdsubj}


The \textit{type-name}s are not limited to the first-class types of
Icon's language definition. The \textit{type-name}s that do not follow
directly from Icon types need further explanation. empty\_type is the
type containing no values and is needed for conveying certain
information to the type inferencing system, such as an unreachable
state. For example, the result type of stop is empty\_type. It may
also be used as the internal type of an empty structure. Contrast this
with null, which consists of the null value.

Variable references are not first-class values in Icon; they cannot be
assigned to variables. However, they do appear in the definition of
Icon as arguments to assignments and as the subject of
dereferencing. For example, the semantics of the expression

{\ttfamily\mdseries
\ \ \ \ \ \ s[3] := s}

\noindent can be described in terms of a substring trapped variable
and a simple variable reference. For this reason, it is necessary to
include these references in the type system of the implementation
language. variable consists of all variable references. It contains
five distinguished subtypes. tvsubs contains all substring trapped
variables. tvtbl contains all table-element trapped variables. kywdint
contains \&random and \&trace. kywdpos contains \&pos. kywdsubj
contains \&subject.

\subsection{Including C Code}

As noted above, C declarations can be included in a declare
clause. Embedded C code may reference these declarations as well as
declarations global to the operation.

Executable C code can be included using one of two actions. 

{\ttfamily\mdseries
\ \ \ \ \ \ \textit{detail-code ::= body \{ extended-C \} {\textbar}}}

{\ttfamily\mdseries
\ \ \ \ \ \ \ \ \ \ \ \ \ \ \ \ \ \ \ \ \ \ \ \ \ \textit{inline \{ extended-C \}}}


body and inline are similar to each other, except that inline
indicates code that is reasonable for the compiler to put in-line when
it can. body indicates that for the in-line version of the operation,
this piece of C code should be put in a separate function in the link
library and the body action should be replaced by a call to that
function. Any parameters or variables from the declare clause needed
by the function must be passed as arguments to the function.
Therefore, it is more efficient to declare variables needed by a body
action within that body than within the declare.  However, the scope
of these local variables is limited to the body action.

Most Icon keywords provide examples of operations that should be
generated in-line. In the following example, nulldesc is a global
variable of type descriptor. It is defined in the include files
automatically included by rtt.

{\ttfamily\mdseries
\ \ \ \&null - the null value.}

{\ttfamily\mdseries
\ \ \ keyword\{1\} null}

{\ttfamily\mdseries
\ \ \ \ \ \ abstract \{}

{\ttfamily\mdseries
\ \ \ \ \ \ \ \ \ return null}

{\ttfamily\mdseries
\ \ \ \ \ \ \ \ \ \}}

{\ttfamily\mdseries
\ \ \ \ \ \ inline \{}

{\ttfamily\mdseries
\ \ \ \ \ \ \ \ \ return nulldesc;}

{\ttfamily\mdseries
\ \ \ \ \ \ \ \ \ \}}

{\ttfamily\mdseries
\ \ \ end}

\subsection{Error Reporting}
{\ttfamily\mdseries
runerr( \textit{msg\_number} [ , descriptor ] ) [ ; ]}


runerr is translated into a call to the run-time error handling
routine. Specifying this as a separate action rather than a C
expression within a body or inline action gives the compiler
additional information about the behavior of the
operation. \textit{msg\_number} is the number used to look up the
error message in a run-time error table. If a descriptor is given, it
is taken to be the offending value.

\subsection{Abstract Type Computations}
{\ttfamily\mdseries
\ \ \ abstract \{ \textit{type-computations} \}}


The behavior of an operation with respect to types is a simplification
of the full semantics of the operation. For example, the semantics of
the function image is to produce the string representing its operand;
its behavior in the type realm is described as simply returning some
string. In general, a good simplification of an operation is too
complicated to be automatically produced from the operation's
implementation (of course, it is always possible to conclude that an
operation can produce any type and can have any side effect, but that
is hardly useful). For this reason, the programmer must use the
abstract action to specify \textit{type-computation}s.

{\ttfamily\mdseries
\ \ \ \textit{type-computations ::= \{ store [ type ] = type [ ; ] \} [ return type [ ; ] ]}}


\textit{type-computations} consist of side effects and a statement of
the result type of the operation. There must be exactly one return
\textit{type} along any path from the start of the operation to C code
containing a return, suspend, or fail.

A side effect is represented as an assignment to the
\textit{store}. The store is analogous to program memory. Program
memory is made up of locations containing values. The store is made up
of locations containing types. A type represents a set of values,
though only certain such sets correspond to types for the purpose of
abstract type computations. Types may be basic types such as all Icon
integers, or they may be composite types such as all Icon integers
combined with all Icon strings. The rules for specifying types are
given below. A location in the store may correspond to one location in
program memory, or it may correspond to several or even an unbounded
number of locations in program memory.  The contents of a location in
the store can be thought of as a conservative (that is, possibly
overestimated) summary of values that might appear in the
corresponding location(s) in program memory at run time.

Program memory can be accessed through a pointer. Similarly, the store
can be indexed by a pointer type, using an expression of the form
store[\textit{type]}, to get at a given location. An Icon global
variable has a location in program memory, and a reference to such a
variable in an Icon program is treated as a pointer to that location.
Similarly, an Icon global variable has a location in the store and,
during type inferencing, a reference to the variable is interpreted as
a pointer type indexing that location in the store. Because types can
be composite, indexing into the store with a pointer type may actually
index several locations. Suppose we have the following side effect

{\ttfamily\mdseries
\ \ \ store[ \textit{type1} ] = type2}


Suppose during type inferencing \textit{type1} evaluates to a
composite pointer type consisting of the pointer types for several
global variables, then all corresponding locations in the store will
be updated. If the above side effect is coded in the assignment
operator, this situation might result from an Icon expression such as

{\ttfamily\mdseries
\ \ \ every (x {\textbar} y) := \&null}


In this example, it is obvious that both variables are changed to the
null type. However, type inferencing can only deduce that at least one
variable in the set is changed. Thus, it must assume that each could
either be changed or left as is. It is only when the left hand side of
the side effect represents a unique program variable that type
inferencing knows that the variable cannot be left as is. In the
current implementation of type inferencing, assignment to a single
named variable is the only side effect where type inferencing
recognizes that the side effect will definitely occur.

Indexing into the store with a non-pointer type corresponds to
assigning to a non-variable. Such an assignment results in error
termination. Type inferencing ignores any non-pointer components in
the index type; they represent execution paths that don't continue and
thus contribute nothing to the types of expressions.

A type in an abstract type computation is of the form 

{\ttfamily\mdseries
\ \ \ \textit{type ::= type-name} {\textbar}}

{\ttfamily\mdseries
\ \ \ \ \ \ \ \ \ \ \ \ \ \ \ type ( variable ) {\textbar}}

{\ttfamily\mdseries
\ \ \ \ \ \ \ \ \ \ \ \ \ \ \ attrb-ref {\textbar}}

{\ttfamily\mdseries
\ \ \ \ \ \ \ \ \ \ \ \ \ \ \ new type-name ( type \{ , type \} ) {\textbar}}

{\ttfamily\mdseries
\ \ \ \ \ \ \ \ \ \ \ \ \ \ \ store [ type ] {\textbar}}

{\ttfamily\mdseries
\ \ \ \ \ \ \ \ \ \ \ \ \ \ \ type ++ type {\textbar}}

{\ttfamily\mdseries
\ \ \ \ \ \ \ \ \ \ \ \ \ \ \ \textit{type ** type} {\textbar}}

{\ttfamily\mdseries
\ \ \ \ \ \ \ \ \ \ \ \ \ \ \ ( type )}


The \textit{type(variable)} expression allows type computations to be
expressed in terms of the type of an argument to an operation. This
must be an unmodified argument. That is, the abstract type computation
involving this expression must not be within the scope of a
conversion. This restriction simplifies the computations needed to
perform type inferencing.

This expression is useful in several contexts, including operations
that deal with structure types. The type system for a program may have
several sub-types for a structure type. The structure types are list,
table, set, record, substring trapped variable, and table-element
trapped variable. Each of these Icon types is a composite type within
the type computations, rather than a basic type. Thus the type
inferencing system may be able to determine a more accurate type for
an argument than can be expressed with a \textit{type-name}. For
example, it is more accurate to use

{\ttfamily\mdseries
\ \ \ if is:list(x) then}

{\ttfamily\mdseries
\ \ \ \ \ \ abstract \{}

{\ttfamily\mdseries
\ \ \ \ \ \ \ \ \ return type(x)}

{\ttfamily\mdseries
\ \ \ \ \ \ \ \ \ \}}

{\ttfamily\mdseries
\ \ \ \ \ \ \textit{actions}}

{\ttfamily\mdseries
\ \ \ else}

{\ttfamily\mdseries
\ \ \ \ \ \ runerr(108, x)}


than it is to use 

{\ttfamily\mdseries
\ \ \ if is:list(x) then}

{\ttfamily\mdseries
\ \ \ \ \ \ abstract \{}

{\ttfamily\mdseries
\ \ \ \ \ \ \ \ \ return list}

{\ttfamily\mdseries
\ \ \ \ \ \ \ \ \ \}}

{\ttfamily\mdseries
\ \ \ \ \ \ actions}

{\ttfamily\mdseries
\ \ \ else}

{\ttfamily\mdseries
\ \ \ \ \ \ runerr(108, x)}


Structure values have internal
{\textasciigrave}{\textasciigrave}structure'{}'. Structure types also
need an internal structure that summarizes the structure of the values
they contain. This structure is implemented with type attributes.
These attributes are referenced using dot notation:

{\ttfamily\mdseries
\ \ \ attrb-ref ::= type . attrb-name}

{\ttfamily\mdseries
\ \ \ attrb-name ::= lst\_elem {\textbar}}

{\ttfamily\mdseries
\ \ \ \ \ \ \ \ \ \ \ \ \ \ \ \ \ \ set\_elem {\textbar}}

{\ttfamily\mdseries
\ \ \ \ \ \ \ \ \ \ \ \ \ \ \ \ \ \ key {\textbar}}

{\ttfamily\mdseries
\ \ \ \ \ \ \ \ \ \ \ \ \ \ \ \ \ \ tbl\_elem {\textbar}}

{\ttfamily\mdseries
\ \ \ \ \ \ \ \ \ \ \ \ \ \ \ \ \ \ default {\textbar}}

{\ttfamily\mdseries
\ \ \ \ \ \ \ \ \ \ \ \ \ \ \ \ \ \ all\_fields {\textbar}}

{\ttfamily\mdseries
\ \ \ \ \ \ \ \ \ \ \ \ \ \ \ \ \ \ str\_var {\textbar}}

{\ttfamily\mdseries
\ \ \ \ \ \ \ \ \ \ \ \ \ \ \ \ \ \ trpd\_tbl}


Just as values internal to structure values are stored in program
memory, types internal to structure types are kept in the store. An
attribute is a pointer type referencing a location in the store.

A list is made up of (unnamed) variables. The lst\_elem attribute of a
list type is a type representing all the variables contained in all
the lists in the type. For example, part of the code for the bang
operator is as follows, where dx is the dereferenced operand.

{\ttfamily\mdseries
\ \ \ type\_case dx of \{}

{\ttfamily\mdseries
\ \ \ \ \ \ list: \{}

{\ttfamily\mdseries
\ \ \ \ \ \ \ \ \ abstract \{}

{\ttfamily\mdseries
\ \ \ \ \ \ \ \ \ \ \ \ return type(dx).lst\_elem}

{\ttfamily\mdseries
\ \ \ \ \ \ \ \ \ \ \ \ \}}

{\ttfamily\mdseries
\ \ \ \ \ \ \ \ \ \textit{actions}}

{\ttfamily\mdseries
\ \ \ \ \ \ \ \ \ \}}

{\ttfamily\mdseries
\ \ \ \ \ \ ...}


This code fragment indicates that, if the argument to bang is in a
list type, bang returns some variable from some list in that type. In
the type realm, bang returns a basic pointer type.

The set\_elem attribute of a set type is similar. The locations of a
set never {\textasciigrave}{\textasciigrave}escape'{}' as
variables. That is, it is not possible to assign to an element of a
set. This is reflected in the fact that a set\_elem is always used as
the index to the store and is never assigned to another location or
returned from an operation. The case in the code from bang for sets is

{\ttfamily\mdseries
\ \ \ set: \{}

{\ttfamily\mdseries
\ \ \ \ \ \ abstract \{}

{\ttfamily\mdseries
\ \ \ \ \ \ \ \ \ return store[type(dx).set\_elem]}

{\ttfamily\mdseries
\ \ \ \ \ \ \ \ \ \}}

{\ttfamily\mdseries
\ \ \ \ \ \ \textit{actions}}

{\ttfamily\mdseries
\ \ \ \ \ \ \}}

Tables types have three attributes. key references a location in the
store containing the type of any possible key value in any table in
the table type. tbl\_elem references a location containing the type of
any possible element in any table in the table type. default
references a location containing the type of any possible default
value for any table in the table type. Only tbl\_elem corresponds to a
variable in Icon. The others must appear as indexes into the store.

Record types are implemented with a location in the store for each
field, but these locations cannot be accessed separately in the type
computations of the implementation language. These are only needed
separately during record creation and field reference, which are
handled as special cases in the compiler. Each record type does have
one attribute, all\_fields, available to type computations. It is a
composite type and includes the pointer types for each of the fields.

Substring trapped variables are implemented as structures. For this
reason, they need structure types to describe them.  The part of the
structure of interest in type inferencing is the reference to the
underlying variable. This is reflected in the one attribute of these
types, str\_var. It is a reference to a location in the store
containing the pointer types of the underlying the variables that are
{\textasciigrave}{\textasciigrave}trapped'{}'. str\_var is only used
as an index into the store; it is never exported from an operation.

Similarly table-element trapped variables need structure types to
implement them. They have one attribute, trpd\_tbl, referencing a
location in the store containing the type of the underlying table. The
key type is not kept separately in the trapped variable type; it must
be immediately added to the table when a table-element trapped
variable type is created. This pessimistically assumes that the key
type will eventually be put in the table, but saves an attribute in
the trapped variable for the key. trpd\_tbl is only used as an index
into the store; it is never exported from an operation.

The type computation, new, indicates that an invocation of the
operation being implemented creates a new instance of a value in the
specified structure type. For example, the implementation of the list
function is

{\ttfamily\mdseries
\ \ \ function\{1\} list(size, initial)}

{\ttfamily\mdseries
\ \ \ \ \ \ abstract \{}

{\ttfamily\mdseries
\ \ \ \ \ \ \ \ \ return new list(type(initial))}

{\ttfamily\mdseries
\ \ \ \ \ \ \ \ \ \}}

{\ttfamily\mdseries
\ \ \ \ \ \ \textit{actions}}

{\ttfamily\mdseries
\ \ \ end}

The type arguments to the new computation specify the initial values
for the \textit{attributes} of the structure. The table type is the
only one that contains multiple attributes. (Note that record
constructors are created during translation and are not specified via
the implementation language.) Table attributes must be given in the
order: key, tbl\_elem, and default.

In the type system for a given program, a structure type is
partitioned into several sub-types (these sub-types are only
distinguished during type inferencing, not at run time). One of these
sub-types is allocated for every easily recognized use of an operation
that creates a new value for the structure type. Thus, the following
Icon program has two list sub-types: one for each invocation of list.

{\ttfamily\mdseries
\ \ \ procedure main()}

{\ttfamily\mdseries
\ \ \ \ \ \ local x}


\bigskip

{\ttfamily\mdseries
\ \ \ \ \ \ x := list(1, list(100))}

{\ttfamily\mdseries
\ \ \ end}

Two operations are available for combining types. Union is denoted by
the operator {\textasciigrave}++' and intersection is denoted by the
operator {\textasciigrave}**'. Intersection has the higher
precedence. These operations interpret types as sets of
values. However, because types may be infinite, these sets are treated
symbolically.

\subsection{C Extensions}

The C code included using the declare, body, and inline actions may
contain several constructs beyond those of standard C. There are five
categories of C extensions: access to interface variables,
declarations, type conversions/type checks, signaling run-time errors,
and return statements.

In addition to their use in the body of an operation, the conversions
and checks, run-time error, and declaration extensions may be used in
ordinary C functions that are put through the implementation language
translator.

\subsection{Interface Variables}

Interface variables include parameters, the identifier for length of
the variable part of an argument list, and the special variable
result. Unconverted parameters, converted parameters with Icon types,
and converted parameters with the internal types tmp\_string and
tmp\_cset are descriptors and within the C code have the type struct
descrip.  Converted parameters with the internal type of C\_integer
have some signed integer type within the C code, but exactly which C
integer type varies between systems. This type has been set up using a
typedef in the automatically included include file so it is available
for use in declarations in C code. Converted parameters with the
internal type of C\_double have the type double within the C
code. Converted parameters of the type C\_string have the type char
. The length of the variable part of a argument list has the type
\textit{int} within the C code.

result is a special descriptor variable. Under some circumstances it
is more efficient to construct a return value in this descriptor than
to use other methods. See Section 5 of the implementation language
reference manual for details.

\subsection{Declarations}

The extension to declarations consists of a new storage class
specifier, tended (register is an example of an existing storage class
specifier). Understanding its use requires some knowledge of Icon
storage management. Only a brief description of storage management is
given here; see the Icon implementation book for further details.

Icon values are represented by descriptors. A descriptor contains both
type information and value information. For large values (everything
other than integers and the null value) the descriptor only contains a
pointer to the value, which resides elsewhere. When such a value is
dynamically created, memory for it is allocated from one of several
memory regions. Strings are allocated from the \textit{string
region}. All other relocatable values are allocated from the
\textit{block region}. The only non-relocatable values are
co-expression stacks and co-expression activation blocks. On some
systems non-relocatable values are allocated in the \textit{static
region}. On other systems there is no static region and these values
are allocated using the C malloc function.

When a storage request is made to a region and there is not enough
room in that region, a \textit{garbage collection} occurs. All
\textit{reachable} values for each region are located. Values in the
string and block regions are moved into a contiguous area at the
bottom of the region, creating (hopefully) free space at the end of
the region.  Unreachable co-expression stacks and activator blocks are
{\textasciigrave}{\textasciigrave}freed'{}'. The garbage collector
must be able to recognize and save all values that might be referenced
after the garbage collection and it must be able to find and update
all pointers to the relocated values. Operation arguments that contain
pointers into one of these regions can always be found by garbage
collection. The implementations of many operations need other
descriptors or pointers into memory regions. The tended storage class
identifies those descriptors and pointers that may have \textit{live}
values when a garbage collection could occur (that is, when a memory
allocation is performed).

A descriptor is implemented as a C struct named descrip, so an example
of a tended descriptor declaration is

{\ttfamily\mdseries
\ \ \ tended struct descrip d;}

Blocks are also implemented as C structs. The following list
illustrates the types of block pointers that may be tended.


{\ttfamily\mdseries
\ \ \ tended struct b\_real *bp;}

{\ttfamily\mdseries
\ \ \ tended struct b\_cset *bp;}

{\ttfamily\mdseries
\ \ \ tended struct b\_file *bp;}

{\ttfamily\mdseries
\ \ \ tended struct b\_proc *bp;}

{\ttfamily\mdseries
\ \ \ tended struct b\_list *bp;}

{\ttfamily\mdseries
\ \ \ tended struct b\_lelem *bp;}

{\ttfamily\mdseries
\ \ \ tended struct b\_table *bp;}

{\ttfamily\mdseries
\ \ \ tended struct b\_telem *bp;}

{\ttfamily\mdseries
\ \ \ tended struct b\_set *bp;}

{\ttfamily\mdseries
\ \ \ tended struct b\_selem *bp;}

{\ttfamily\mdseries
\ \ \ tended struct b\_record *bp;}

{\ttfamily\mdseries
\ \ \ tended struct b\_tvkywd *bp;}

{\ttfamily\mdseries
\ \ \ tended struct b\_tvsubs *bp;}

{\ttfamily\mdseries
\ \ \ tended struct b\_tvtbl *bp;}

{\ttfamily\mdseries
\ \ \ tended struct b\_refresh *bp;}

{\ttfamily\mdseries
\ \ \ tended struct b\_coexpr *cp;}


Alternatively, a union pointer can be used to tend a pointer to any
kind of block.

{\ttfamily\mdseries
\ \ \ tended union block *bp;}

Character pointers may also be tended. However, garbage collection
needs a length associated with a pointer into the string
region. Unlike values in the block region, the strings themselves do
not have a length stored with them. Garbage collection treats a tended
character pointer as a zero-length string. These character pointers
are almost always pointers into some string, so garbage collection
effectively treats them as zero-length substrings of the strings. The
string as a whole must be tended by some descriptor so that it is
preserved. The purpose of tending a character pointer is to insure
that the pointer is relocated with the string it points into. An
example is

{\ttfamily\mdseries
\ \ \ tended char *s1, *s2;}

Tended arrays are not supported. tended may only be used with
variables of local scope. tended and register are mutually
exclusive. If no initial value is given, one is supplied that is
consistent with garbage collection.

\subsection{Type Conversions/Type Checks}

Some conditional expressions have been added to C. These are based on
type checks in the type specification part of the implementation
language.

{\ttfamily\mdseries
\ \ \ is: \textit{type-name} ( source )}

{\ttfamily\mdseries
\ \ \ cnv: dest-type ( source , destination )}

{\ttfamily\mdseries
\ \ \ def: dest-type ( source , value , destination )}


\textit{source} must be an Icon value, that is, a
descriptor. \textit{destination} must be a variable whose type is
consistent with the conversion. These type checks may appear anywhere
a conditional expression is valid in a C program.  Note that is, cvn,
and def are reserved words to distinguish them from labels.

The type\_case statement may be used in extended C. This statement has
the same form as the corresponding action, but in this context, C code
replaces the \textit{action}s in the \textit{type-select} clauses.

\subsection{Signaling Run-time Errors}

runerr is used for signaling run-time errors. It acts like a function
but may take either 1 or 2 arguments. The first argument is the error
number. If the error has an associated value, the second argument is a
descriptor containing that value.

\subsection{Return Statements}

There are three statements for leaving the execution of an
operation. These are analogous to the corresponding expressions in the
Icon language.

{\ttfamily\mdseries
\ \ \ \textit{ret-statments} ::= return ret-value ; {\textbar}}

{\ttfamily\mdseries
\ \ \ \ \ \ \ \ \ \ \ \ \ \ \ \ \ \ \ \ \ suspend ret-value ; {\textbar}}

{\ttfamily\mdseries
\ \ \ \ \ \ \ \ \ \ \ \ \ \ \ \ \ \ \ \ \ fail ;}

{\ttfamily\mdseries
\ \ \ ret-value ::= descriptor {\textbar}}

{\ttfamily\mdseries
\ \ \ \ \ \ \ \ \ \ \ \ \ \ \ \ \ C\_integer expression {\textbar}}

{\ttfamily\mdseries
\ \ \ \ \ \ \ \ \ \ \ \ \ \ \ \ \ C\_double expression {\textbar}}

{\ttfamily\mdseries
\ \ \ \ \ \ \ \ \ \ \ \ \ \ \ \ \ C\_string expression {\textbar}}

{\ttfamily\mdseries
\ \ \ \ \ \ \ \ \ \ \ \ \ \ \ \ \ descript-constructor}


\textit{descriptor} is an expression of type struct descrip. For example 

{\ttfamily\mdseries
\ \ \ \{}

{\ttfamily\mdseries
\ \ \ \ \ \ tended struct descrip dp;}

{\ttfamily\mdseries
\ \ \ \ \ \ ...}

{\ttfamily\mdseries
\ \ \ \ \ \ suspend dp;}

{\ttfamily\mdseries
\ \ \ \ \ \ ...}

{\ttfamily\mdseries
\ \ \ \}}


Use of C\_integer, C\_double, or C\_string to prefix an expression
indicates that the expression evaluates to the indicated C type and
not to a descriptor. When necessary, a descriptor is constructed from
the result of the expression, but when possible the Icon compiler
produces code that can use the raw C value (See Section 5 of the
implementation language reference manual). As an example, the integer
case in the divide operation is simply

{\ttfamily\mdseries
\ \ \ inline \{}

{\ttfamily\mdseries
\ \ \ \ \ \ return C\_integer x / y;}

{\ttfamily\mdseries
\ \ \ \ \ \ \}}


Note that a returned C string must not be in a local (dynamic)
character array; it must have a global lifetime.

A \textit{descript-constructor} is an expression that explicitly
converts a pointer into a descriptor. It is only valid in a return
statement, because it builds the descriptor in the implicit location
of the return value.

{\ttfamily\mdseries
\ \ \ \textit{descript-constructor} ::= string ( length , char-ptr ) {\textbar}}

{\ttfamily\mdseries
\ \ \ \ \ \ \ \ \ \ \ \ \ \ \ \ \ \ \ \ \ \ \ \ \ \ \ \ cset ( block-ptr ) {\textbar}}

{\ttfamily\mdseries
\ \ \ \ \ \ \ \ \ \ \ \ \ \ \ \ \ \ \ \ \ \ \ \ \ \ \ \ real ( block-ptr ) {\textbar}}

{\ttfamily\mdseries
\ \ \ \ \ \ \ \ \ \ \ \ \ \ \ \ \ \ \ \ \ \ \ \ \ \ \ \ file ( block-ptr ) {\textbar}}

{\ttfamily\mdseries
\ \ \ \ \ \ \ \ \ \ \ \ \ \ \ \ \ \ \ \ \ \ \ \ \ \ \ \ procedure ( block-ptr ) {\textbar}}

{\ttfamily\mdseries
\ \ \ \ \ \ \ \ \ \ \ \ \ \ \ \ \ \ \ \ \ \ \ \ \ \ \ \ list ( block-ptr ) {\textbar}}

{\ttfamily\mdseries
\ \ \ \ \ \ \ \ \ \ \ \ \ \ \ \ \ \ \ \ \ \ \ \ \ \ \ \ set ( block-ptr ) {\textbar}}

{\ttfamily\mdseries
\ \ \ \ \ \ \ \ \ \ \ \ \ \ \ \ \ \ \ \ \ \ \ \ \ \ \ \ record ( block-ptr ) {\textbar}}

{\ttfamily\mdseries
\ \ \ \ \ \ \ \ \ \ \ \ \ \ \ \ \ \ \ \ \ \ \ \ \ \ \ \ table ( block-ptr ) {\textbar}}

{\ttfamily\mdseries
\ \ \ \ \ \ \ \ \ \ \ \ \ \ \ \ \ \ \ \ \ \ \ \ \ \ \ \ co\_expression ( stack-ptr ) {\textbar}}

{\ttfamily\mdseries
\ \ \ \ \ \ \ \ \ \ \ \ \ \ \ \ \ \ \ \ \ \ \ \ \ \ \ \ tvtbl ( block-ptr ) {\textbar}}

{\ttfamily\mdseries
\ \ \ \ \ \ \ \ \ \ \ \ \ \ \ \ \ \ \ \ \ \ \ \ \ \ \ \ named\_var ( descr-ptr ) {\textbar}}

{\ttfamily\mdseries
\ \ \ \ \ \ \ \ \ \ \ \ \ \ \ \ \ \ \ \ \ \ \ \ \ \ \ \ struct\_var ( descr-ptr , block-ptr ) {\textbar}}

{\ttfamily\mdseries
\ \ \ \ \ \ \ \ \ \ \ \ \ \ \ \ \ \ \ \ \ \ \ \ \ \ \ \ substr ( descr-ptr , start , len ) {\textbar}}

{\ttfamily\mdseries
\ \ \ \ \ \ \ \ \ \ \ \ \ \ \ \ \ \ \ \ \ \ \ \ \ \ \ \ kywdint ( descr-ptr ) {\textbar}}

{\ttfamily\mdseries
\ \ \ \ \ \ \ \ \ \ \ \ \ \ \ \ \ \ \ \ \ \ \ \ \ \ \ \ kywdpos ( descr-ptr ) {\textbar}}

{\ttfamily\mdseries
\ \ \ \ \ \ \ \ \ \ \ \ \ \ \ \ \ \ \ \ \ \ \ \ \ \ \ \ kywdsubj ( descr-ptr )}


The arguments to string are the length of the string and the pointer
to the start of the string. \textit{block-ptr}s are pointers to blocks
of the corresponding types. \textit{stack-ptr} is a pointer to a
co-expression stack.  \textit{descr-ptr} is a pointer to a
descriptor. named\_var is used to create a reference to a variable
(descriptor) that is not in a block. struct\_var is used to create a
reference to a variable that is in a block. The Icon garbage collector
works in terms of whole blocks. It cannot preserve just a single
variable in the block, so the descriptor referencing a variable must
contain enough information for the garbage collector to find the start
of the block. That is what the \textit{block-ptr} is for. substr
creates a substring trapped variable for the given descriptor,
starting point within the string, and length. kywdint, kywdpos, and
kywdsubj create references to keyword variables.

Note that returning either C\_double \textit{expression} or
substr(\textit{descr-ptr, start, len}) may trigger a garbage
collection. \newline


\bigskip


\bigskip


\bigskip

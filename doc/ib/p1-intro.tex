\chapter{Introduction}
\label{Intro-Chapter}

\textsc{Perspective}: The implementation of complex software systems
is a fascinating subject --- and an important one. Its theoretical and
practical aspects occupy the attention and energy of many persons, and
it consumes vast amounts of computational resources. In general terms,
it is a broad subject ranging from operating systems to programming
languages to data-base systems to real-time control systems, and so
on.


Past work in these areas has resulted in an increasingly better
understanding of implementation techniques, more sophisticated and
efficient systems, and tools for automating various aspects of
software production. Despite these advances, the implementation of
complex software systems remains challenging and exciting. The
problems are difficult, and every advance in the state of the art
brings new and more difficult problems within reach.


Part I of this book addresses a very small portion of the problem of
implementing complex software systems{}---the implementation of a very
high-level programming language that is oriented toward the
manipulation of structures and strings of characters.


In a narrow sense, this book describes an implementation of a specific
programming language, Icon. In a broader sense, it deals with a
language-design philosophy, an approach to implementation, and
techniques that apply to the implementation of many programming
languages as well as related types of software systems.


The focus of this book is the implementation of programming language
features that are at a high conceptual level{}---features that are
easy for human beings to use as opposed to features that fit
comfortably on conventional computer architectures. The orientation of
the implementation is generality and flexibility, rather than maximum
efficiency of execution. The problem domain is strings and structures
rather than numbers. It is these aspects that set the implementation
of Icon apart from more conventional programming-language
implementations.

\section{Implementing Programming Languages}

In conventional programming languages, most of the operations that are
performed when a program is executed can be determined, statically, by
examining the text of the program. In addition, the operations of most
programming languages have a fairly close correspondence to the
architectural characteristics of the computers on which they are
implemented.  When these conditions are met, source-code constructs
can be mapped directly into machine instructions for the computer on
which they are to be executed. The term \textit{compilation} is used
for this translation process, and most persons think of the
implementation of a programming language in terms of a compiler.

Writing a compiler is a complex and difficult task that requires
specialized training, and the subject of compilation has been studied
extensively (Waite and Goos, 1984; Aho, Lam, Sethi and Ullman
2006). Most of the issues of data representation and code generation
are comparatively well understood, and there are now many tools for
automating portions of the compiler-writing task (Lesk 1975, Johnson
1975).


In addition to the compiler proper, an implementation of a programming
language usually includes a run-time component that contains
subroutines for performing computations that are too complex to
compile in-line, such as input, output, and mathematical functions.


Some programming languages have features whose meanings cannot be
determined statically from the text of a source-language program, but
which may change during program execution. Such features include
changes in the meaning of functions during execution, the creation of
new data types at run time, and self-modifying programs. Some
programming languages also have features, such as pattern matching,
that do not have correspondences in the architecture of conventional
computers. In such cases, a compiler cannot translate the source
program directly into executable code.  Very high-level operations,
such as pattern matching, and features like automatic storage
management significantly increase the importance and complexity of the
run-time system. For languages with these characteristics-{}-languages
such as APL, LISP, SNOBOL4, SETL, Prolog, and Icon-much of the
substance of the implementation is in the run-time system rather than
in translation done by a compiler. While compiler writing is
relatively well understood, run-time systems for most programming
languages with dynamic features and very high-level operations are
not.


Programming languages with dynamic aspects and novel features are
likely to become more important rather than less important. Different
problems benefit from different linguistic mechanisms. New
applications place different values on speed of execution, memory
requirements, quick solutions, programmer time and talent, and so
forth. For these reasons, programming languages continue to
proliferate. New programming languages, by their nature, introduce new
features.


All of this creates difficulties for the implementer. Less of the
effort involved in implementations for new languages lies in the
comparatively familiar domain of compilation and more lies in new and
unexplored areas, such as pattern matching and novel
expression-evaluation mechanisms.

The programming languages that are the most challenging to implement
are also those that differ most from each other.  Nevertheless, there
are underlying principles and techniques that are generally
applicable, and existing implementations contain many ideas that can
be used or extended in new implementations.

\section{The Background for Icon}

Before describing the Icon programming language and its
implementation, some historical context is needed, since both the
language and its implementation are strongly influenced by earlier
work.

\index{SNOBOL programming languages|(}
Icon has its roots in a series of programming languages that bear the
name SNOBOL. The first SNOBOL language was conceived and implemented
in the early 1960s at Bell Telephone Laboratories in response to the
need for a programming tool for manipulating strings of characters at
a high conceptual level (Farber, Griswold, and Polonsky 1964). It
emphasized ease of programming at the expense of efficiency of
execution; the programmer was considered to be a more valuable
resource than the computer.

This rather primitive language proved to be popular, and it was
followed by successively more sophisticated languages: SNOBOL2,
SNOBOL3 (Farber, Griswold, and Polonsky 1966), and finally SNOBOL4
(Griswold, Poage, and Polonsky 1971).  Throughout the development of
these languages, the design emphasis was on ease of programming rather
than on ease of implementation (Griswold 1981). Potentially valuable
features were not discarded because they might be inefficient or
difficult to implement. The aggressive pursuit of this philosophy led
to unusual language features and to challenging implementation
problems.

SNOBOL4 still is in wide use. Considering its early origins, some of
its facilities are remarkably advanced. It features a pattern-matching
facility with backtracking control structures that effectively
constitutes a sublanguage. SNOBOL4 also has a variety of data
structures, including tables with associative lookup. Functions and
operators can be defined and redefined during program execution.
Identifiers can be created at run-time, and a program can even modify
itself by means of run-time compilation.

Needless to say, SNOBOL4 is a difficult language to implement, and
most of the conventional compilation techniques have little
applicability to it. Its initial implementation was, nonetheless,
sufficiently successful to make SNOBOL4 widely available on machines
ranging from large mainframes to personal computers (Griswold
1972). Subsequent implementations introduced a variety of clever
techniques and fast, compact implementations (Santos 1971; Gimpel
1972a; Dewar and McCann 1977). The lesson here is that the design of
programming languages should not be overly inhibited by perceived
implementation problems, since new implementation techniques often can
be devised to solve such problems effectively and efficiently.

It is worth noting that the original implementation of SNOBOL4 was
carried out concomitantly with language design. The implementation was
sufficiently flexible to serve as a research tool in which
experimental language features could be incorporated easily and tested
before they were given a permanent place in the language.

Work on the SNOBOL languages continued at the University of Arizona in
the early 1970s. In 1975, a new language, called SL5
({\textquotedbl}SNOBOL Language 5{\textquotedbl}), was developed to
allow experimentation with a wider variety of programming-language
constructs, especially a sophisticated procedure mechanism (Griswold
and Hanson, 1977; Hanson and Griswold 1978). SL5 extended earlier work
in pattern matching, but pattern matching remained essentially a
sublanguage with its own control structures, separate from the rest of
the language.
\index{SNOBOL programming languages|)}

The inspiration for Icon came in 1976 with a realization that the
control structures that were so useful in pattern matching could be
integrated with conventional computational control structures to yield
a more coherent and powerful programming language.


The first implementation of Icon (Griswold and Hanson 1979) was
written in Ratfor, a preprocessor for Fortran that supports structured
programming features (Kernighan 1975). Portability was a central
concern in this implementation.  The implementation of Icon described
in this book is a successor to that first implementation. It borrows
much from earlier implementations of SNOBOL4, SL5, and the Ratfor
implementation of Icon. As such, it is a distillation and refinement
of implementation techniques that have been developed over a period of
more than twenty years.


\bigskip

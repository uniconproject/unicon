%-------------------------------------------------------------------------------
% Macros for the Icon Implementation Book LaTeX pictures
%
%    Don Ward
%    December 2016
%
%  The macros are designed so that if the same starting coord is used,
%  the components of the box come out in the right place. (x,y) is
%  the location of the bottom left-hand corner of the box.
%  eg.
%
% \put(x,y){\dvbox{title}{flags}{}}% Note that #3 (value) is null because we put a rdptr there
% \put(x,y){\leftboxlabels{top-left}{bottom-left}}
% \put(x,y){\trboxlabel{top-right-label}}
% \put(x,y){\rdptr{30}{20}}
% \put(x,y){\upetc}
%
% produces a picture (roughly) like this
%
%                                 .
%                                 .
%                  |              .               |
%                  |------------------------------|
%       top-left   | flags                  title |   top-right-label
%                  |..............................|
%    bottom-left   |                      --------+------------
%                  |------------------------------|           |
%              (x,y)                                          |
%                                                             V
%
%   The boxes are mostly 100pt wide and 32pt high, although some are 48pt high
%   and there are three  half-height (16pt) boxes, for use when there is an odd
%   number of 'slots' in the picture).
%-------------------------------------------------------------------------------
% Draw a descriptor/value box  \dvbox{title}{flags}{value}
%                  |------------------------------|
%                  | flags                  title |
%                  |..............................|
%                  |                        value |
%                  |------------------------------|
\newcommand{\dvbox}[3]{%
\begin{picture}(120,32)%
\multiput(0,16)(2,0){50}{\line(1,0){1}}% The dotted line in the middle
\put(0,0){\framebox(100,32){}}%          The box
\put(-2,-4){\makebox(100,32)[tr]{#1}}%   title
\put(2,-4){\makebox(100,32)[tl]{#2}}%    flags
\put(-2,4){\makebox(100,32)[br]{#3}}%    value
\end{picture}%
}%
%-------------------------------------------------------------------------------
% Draw a descriptor/value box with an arrow \dvboxptr{title}{flags}{len}{value}
% (len is the length of the arrow; value is placed after it, to the right)
%                  |------------------------------|
%                  | flags                  title |
%                  |..............................|
%                  |                       -------+---------------->  value
%                  |------------------------------|
\newcommand{\dvboxptr}[4]{%
\begin{picture}(120,32)%
\multiput(0,16)(2,0){50}{\line(1,0){1}}% The dotted line in the middle
\put(0,0){\framebox(100,32){}}%          The box
\put(-2,-4){\makebox(100,32)[tr]{#1}}%    title
\put(2,-4){\makebox(100,32)[tl]{#2}}%     flags
\put(80,8){\vector(1,0){#3}}%             arrow
\put(90,6){\hspace{#3pt}#4}%              value
\end{picture}%
}%
%-------------------------------------------------------------------------------
% Draw a block box with a single value \blkbox{title}{value_1}
% A block box is like a dvbox, but no flags and a solid separator line
\newcommand{\blkbox}[2]{%
\begin{picture}(120,32)%
\put(0,16){\line(1,0){100}}%             The line in the middle
\put(0,0){\framebox(100,32){}}%          The box
\put(-2,-4){\makebox(100,32)[tr]{#1}}%    title
\put(-2,4){\makebox(100,32)[br]{#2}}%     value
\end{picture}%
}%
%-------------------------------------------------------------------------------
% Draw a block box with a single value and an arrow \bklptrkbox{title}{len}{value}
\newcommand{\blkboxptr}[3]{%
\begin{picture}(120,32)%
\put(0,16){\line(1,0){100}}%             The line in the middle
\put(0,0){\framebox(100,32){}}%          The box
\put(-2,-4){\makebox(100,32)[tr]{#1}}%   title
\put(80,8){\vector(1,0){#2}}%            arrow
\put(90,6){\hspace{#2pt}#3}%             value
\end{picture}%
}%
%-------------------------------------------------------------------------------
% Draw a block box with two arrows and values \dblptrkbox{len}{value1}{value2}
\newcommand{\dblkboxptr}[3]{%
\begin{picture}(120,32)%
\put(0,16){\line(1,0){100}}%             The line in the middle
\put(0,0){\framebox(100,32){}}%          The box
\put(80,24){\vector(1,0){#1}}%           arrow
\put(90,22){\hspace{#1pt}#2}%            value1
\put(80,8){\vector(1,0){#1}}%            arrow
\put(90,6){\hspace{#1pt}#3}%             value2
\end{picture}%
}%
%-------------------------------------------------------------------------------
% Draw a block box with two values \blkboxtwo{title}{value_1}{value_2}
% This box is 48pt high, rather than 32
\newcommand{\blkboxtwo}[3]{%
\begin{picture}(120,48)%
\put(0,16){\line(1,0){100}}%             Lower separator
\put(0,32){\line(1,0){100}}%             Upper separator
\put(0,0){\framebox(100,48){}}%          The box
\put(-2,-4){\makebox(100,48)[tr]{#1}}%   title
\put(-2,-4){\makebox(100,32)[tr]{#2}}%   value_1
\put(-2,4){\makebox(100,48)[br]{#3}}%    value_2
\end{picture}%
}%
%-------------------------------------------------------------------------------
% Draw a block box with a large value \blklrgbox{title}{value}
\newcommand{\blklrgbox}[2]{%
\begin{picture}(120,48)%
\put(0,32){\line(1,0){100}}%             Separator
\multiput(0,16)(95,0){2}{\line(1,0){5}}% Decorative flim-flam
\put(0,0){\framebox(100,48){}}%          The box
\put(-2,-4){\makebox(100,48)[tr]{#1}}%   title
\put(0,0){\makebox(100,32){#2}}%         value
\end{picture}%
}%
%-------------------------------------------------------------------------------
% Half height boxes
%
% Draw a word box (half height) \wordbox{title}{flags}
\newcommand{\wordbox}[2]{%
\begin{picture}(120,16)%
\put(0,0){\framebox(100,16){}}%          The box
\put(-2,-4){\makebox(100,16)[tr]{#1}}%   title
\put(2,-4){\makebox(100,16)[tl]{#2}}%    flags
\end{picture}%
}%
%-------------------------------------------------------------------------------
% Draw a word box (half height) with an arrow \wordboxptr{len}{value}
\newcommand{\wordboxptr}[2]{%
\begin{picture}(120,16)%
\put(0,0){\framebox(100,16){}}%          The box
\put(80,8){\vector(1,0){#1}}%             arrow
\put(90,6){\hspace{#1pt}#2}%              value
\end{picture}%
}%
%-------------------------------------------------------------------------------
% Draw a word box (half height) with a null pointer symbol and a label
% \nullptrbox{label}
\newcommand{\nullptrbox}[1]{%
\begin{picture}(120,16)%
\put(0,0){\framebox(100,16){}}%          The box
\put(-2,4){\makebox(100,32)[br]{\o}}%    Null ptr symbol
\put(10,8){\makebox(100,32)[br]{\makebox(0,0)[l]{#1}}}% label
\end{picture}%
}%
%-------------------------------------------------------------------------------
% Draw a vertical stack of n words   \wordpile{n}{\nullptrbox{}}
\newcommand{\wordpile}[2]{%
\begin{picture}(120,16)% A porky: it's really (120, 16*n)
\multiput(0,0)(0,16){#1}{#2}%
\end{picture}%
}%
%-------------------------------------------------------------------------------
% Draw a vertical stack of n dwords   \dwordpile{n}{\dvbox{null}{n}{value}}
\newcommand{\dwordpile}[2]{%
\begin{picture}(120,32)% A porky: it's really (120, 32*n)
\multiput(0,0)(0,32){#1}{#2}%
\end{picture}%
}%
%-------------------------------------------------------------------------------
% Box labels
% Draw two right-aliged left hand labels  \leftboxlabels{top-left}{bottom-left}
\newcommand{\leftboxlabels}[2]{%
\begin{picture}(120,32)%
\put(-10,-8){\makebox(100,32)[tl]{\makebox(0,0)[r]{#1}}}%
\put(-10,8){\makebox(100,32)[bl]{\makebox(0,0)[r]{#2}}}%
\end{picture}%
}%
% Draw two left-aliged right hand labels  \rightboxlabels{top-right}{bottom-right}
\newcommand{\rightboxlabels}[2]{%
\begin{picture}(120,32)%
\put(10,-8){\makebox(100,32)[tr]{\makebox(0,0)[l]{#1}}}%
\put(10,8){\makebox(100,32)[br]{\makebox(0,0)[l]{#2}}}%
\end{picture}%
}%
%-------------------------------------------------------------------------------
% Draw individual labels (tl = top left, br = bottom right etc.)
\newcommand{\tlboxlabel}[1]{%
\begin{picture}(120,32)%
\put(-10,-8){\makebox(100,32)[tl]{\makebox(0,0)[r]{#1}}}%
\end{picture}%
}%
\newcommand{\blboxlabel}[1]{%
\begin{picture}(120,32)%
\put(-10,8){\makebox(100,32)[bl]{\makebox(0,0)[r]{#1}}}%
\end{picture}%
}%
\newcommand{\trboxlabel}[1]{%
\begin{picture}(120,32)%
\put(10,-8){\makebox(100,32)[tr]{\makebox(0,0)[l]{#1}}}%
\end{picture}%
}%
\newcommand{\brboxlabel}[1]{%
\begin{picture}(120,32)%
\put(10,8){\makebox(100,32)[br]{\makebox(0,0)[l]{#1}}}%
\end{picture}%
}%
%-------------------------------------------------------------------------------
% Arrows (from the v-word)
\newcommand{\rptr}[1]{%  An arrow (of length #1) to the right
\begin{picture}(120,32)%
\put(80,8){\vector(1,0){#1}}%
\end{picture}%
}%
\newcommand{\lptr}[1]{%  An arrow (of length #1) to the left
\begin{picture}(120,32)%
\put(20,8){\vector(-1,0){#1}}%
\end{picture}%
}%
% Arrows with a right-angle bend (ru = right then up, ld = left then down etc.)
\newcommand{\rdptr}[2]{%
\begin{picture}(120,32)%
\put(80,8){\line(1,0){#1}}%
\put(80,8){\hspace{#1pt}\vector(0,-1){#2}}%
\end{picture}%
}%
\newcommand{\ruptr}[2]{%
\begin{picture}(120,32)%
\put(80,8){\line(1,0){#1}}%
\put(80,8){\hspace{#1pt}\vector(0,1){#2}}%
\end{picture}%
}%
\newcommand{\ldptr}[2]{%
\begin{picture}(120,32)%
\put(20,8){\line(-1,0){#1}}%
\put(20,8){\hspace{-#1pt}\vector(0,-1){#2}}%
\end{picture}%
}%
\newcommand{\luptr}[2]{%
\begin{picture}(120,32)%
\put(20,8){\line(-1,0){#1}}%
\put(20,8){\hspace{-#1pt}\vector(0,1){#2}}%
\end{picture}%
}%
%-------------------------------------------------------------------------------
% Register pointers (to the vword)
\newcommand{\lregptr}[2]{%
\begin{picture}(32,32)%
\put(-2,13){\makebox(0,0)[r]{\makebox(0,0)[r]{#1}\hspace{6pt}\vector(1,0){#2}}}%
\end{picture}%
}%
\newcommand{\rregptr}[2]{%
\begin{picture}(32,32)%
\put(102,8){\makebox(0,0)[l]{\hspace{#2pt}\vector(-1,0){#2}}}%
\put(98,12){\makebox(0,0)[l]{\hspace{#2pt}\hspace{10pt}{\makebox(0,-10)[l]{#1}}}}%
\end{picture}%
}%
% Register pointers (to the dword)
\newcommand{\luregptr}[2]{%
\begin{picture}(32,32)%
\put(-2,29){\makebox(0,0)[r]{\makebox(0,0)[r]{#1}\hspace{6pt}\vector(1,0){#2}}}%
\end{picture}%
}%
\newcommand{\ruregptr}[2]{%
\begin{picture}(32,32)%
\put(102,24){\makebox(0,0)[l]{\hspace{#2pt}\vector(-1,0){#2}}}%
\put(98,28){\makebox(0,0)[l]{\hspace{#2pt}\hspace{10pt}{\makebox(0,-10)[l]{#1}}}}%
\end{picture}%
}%
%-------------------------------------------------------------------------------
% vertical dots and bars
\newcommand{\downetc}{% NB. This picture goes BELOW it's origin
\begin{picture}(120,16)(0,0)%
\put(0,0){\line(1,0){100}}%
\put(0,0){\line(0,-1){16}}%
\put(100,0){\line(0,-1){16}}%
\put(48.5,-8){.}%
\put(48.5,-16){.}%
\put(48.5,-24){.}%
\end{picture}%
}%
\newcommand{\upetc}{% NB this picture is offset up by 32
\begin{picture}(120,16)(0,-32)%
\put(0,0){\line(1,0){100}}%
\put(0,0){\line(0,1){16}}%
\put(100,0){\line(0,1){16}}%
\put(48.5,8){.}%
\put(48.5,16){.}%
\put(48.5,24){.}%
\end{picture}%
}%
\newcommand{\downbars}{% NB. This picture goes BELOW it's origin
\begin{picture}(120,16)%
\put(0,0){\line(1,0){100}}%
\put(0,0){\line(0,-1){16}}%
\put(100,0){\line(0,-1){16}}%
\end{picture}%
}%
\newcommand{\upbars}{% NB this picture is offset up by 32
\begin{picture}(120,16)(0,-32)%
\put(0,0){\line(1,0){100}}%
\put(0,0){\line(0,1){16}}%
\put(100,0){\line(0,1){16}}%
\end{picture}%
}%
\newcommand{\updownbars}{% NB. This picture goes BELOW it's origin
\begin{picture}(120,16)%
\put(0,0){\line(1,0){100}}%
\put(0,0){\line(0,-1){16}}%
\put(100,0){\line(0,-1){16}}%
\put(0,0){\line(0,1){16}}%
\put(100,0){\line(0,1){16}}%
\end{picture}%
}%
% ----------------------------------------------------------------------
% Draw an oval around centred text
%      \put(x,y){\ovt{len}{height}{text}}
% puts an oval of the specified height and length centred at (x,y)
% It would be nice to calculate len based on the length of the text,
% but my TeX/LaTeX-fu isn't up to it.
\newcommand{\ovt}[3]{%
{\makebox(0,0){%
{\makebox(0,#2){#3}}%
{\makebox(0,#2)[b]{\oval(#1,#2)}}%
}}}%
%
% ----------------------------------------------------------------------
% Stuff to test alignment:
%
%   set param2 to ten times param 1
%   Ugly, but functional
\newcommand{\tentimes}[2]{%
\setcounter{#2}{#1}%
\addtocounter{#2}{#1}\addtocounter{#2}{#1}\addtocounter{#2}{#1}%
\addtocounter{#2}{#1}\addtocounter{#2}{#1}\addtocounter{#2}{#1}%
\addtocounter{#2}{#1}\addtocounter{#2}{#1}\addtocounter{#2}{#1}%
}%
%
%-----  \graphpaper{{x-times}{y-times}
%
% Draw a 10pt grid with 1pt units. 
%    x-times is how many horizontal 10x10 squares
%    y-times is how many vertical 10x10 squares
%
% Useful for calculating line position/length and for checking alignment.
% Best drawn before actual picture, so the picture goes on top of the graph.
\newcounter{ngpx}% 
\newcounter{ngpy}%
\newcommand{\graphpaper}[2]{%
\begin{picture}(0,0)%
{%
\tentimes{#1}{ngpx}%  length of horizontal lines
\tentimes{#2}{ngpy}%  length of vertical lines
\color[rgb]{0.8,0.8,0.8}%                 units grey
\multiput(0,0)(10,0){#1}{%                draw 9 vertical lines
\multiput(1,0)(1,0){9}{\line(0,1){\value{ngpy}}}%   at (1,0) (2,0) (3,0) ...
}%
\multiput(0,0)(0,10){#2}{%                draw 9 horizontal lines
\multiput(0,1)(0,1){9}{\line(1,0){\value{ngpx}}}%   at (0,1) (0,2),(0,3) ...
}%
\color[rgb]{0.5,0.5,0.5}%                 tens grey (darker)
\multiput(0,0)(10,0){#1}{\put(0,0){\line(0,1){\value{ngpy}}}}% vertical lines every (10,0)
\multiput(0,0)(0,10){#2}{\put(0,0){\line(1,0){\value{ngpx}}}}% horizontal lines every (0,10)
}%
\end{picture}%
}%
%% Old version (with four parameters \graphpaper{x-len}{y-len}{x-times}{y-times})
%% \newcommand{\graphpaper}[4]{%
%% \begin{picture}(0,0)%
%% {%
%% \color[rgb]{0.8,0.8,0.8}%                 units grey
%% \multiput(0,0)(10,0){#3}{%                draw 9 vertical lines
%% \multiput(1,0)(1,0){9}{\line(0,1){#2}}%   at (1,0) (2,0) (3,0) ...
%% }%
%% \multiput(0,0)(0,10){#4}{%                draw 9 horizontal lines
%% \multiput(0,1)(0,1){9}{\line(1,0){#1}}%   at (0,1) (0,2),(0,3) ...
%% }%
%% \color[rgb]{0.5,0.5,0.5}%                 tens grey (darker)
%% \multiput(0,0)(10,0){#3}{\put(0,0){\line(0,1){#2}}}% vertical lines every (10,0)
%% \multiput(0,0)(0,10){#4}{\put(0,0){\line(1,0){#1}}}% horizontal lines every (0,10)
%% }%
%% \end{picture}%
%% }%
%
%----- Single cross-hair   \put(x,y){\crosshair}
\newcommand{\crosshair}{%
{\color{blue}%
\put(0,-10){\vector(0,1){9.5}}%
\put(-10,0){\vector(1,0){9.5}}%
\put(10,0){\vector(-1,0){9.5}}%
\put(0,10){\vector(0,-1){9.5}}%
}% color
}%
%----- Test Grid of 8 "cross-hairs. 
% Parameters are width (usually 100)
% and depth (usually 16)
\newcommand{\crosshairs}[2]{%
\multiput(0,0)(#1,0){2}{\multiput(0,0)(0,#2){4}{\crosshair}}%
}%
%
%-------- test picture
%% \noindent\begin{picture}(400,200)
%% \put(0,0){\graphpaper{400}{200}{40}{20}}
%% \put(20,20){\crosshairs{100}{16}}
%% \begin{picture}(0,0)(-20,-20)
%% \put(0,0){\dvbox{title}{flags}{value}}
%% \put(0,0){\dvboxptr{title}{flags}{40}{stuff}}
%% \put(0,0){\blkbox{title}{value}}
%% \put(0,0){\blkboxptr{title}{40}{value}}
%% \put(0,0){\dblkboxptr{40}{V1}{V2}}
%% \put(0,0){\blkboxtwo{title}{$value_1$}{$value_2$}}
%% \put(0,0){\blklrgbox{title}{value}}
%% \put(0,0){\wordbox{value}}
%% \put(0,32){\wordboxptr{40}{value}}
%% \put(0,0){\nullptrbox{label}}
%% \put(0,32){\leftboxlabels{top-left}{bottom-left}}
%% \put(0,32){\rightboxlabels{top-right}{bottom-right}}
%% \put(0,0){\tlboxlabel{top-left}}
%% \put(0,0){\blboxlabel{bottom-left}}
%% \put(0,0){\trboxlabel{top-right}}
%% \put(0,0){\brboxlabel{bottom-right}}
%% \put(0,16){\rptr{40}}
%% \put(0,16){\lptr{40}}
%% \put(0,32){\rdptr{40}{16}}
%% \put(0,32){\ldptr{40}{16}}
%% \put(0,0){\ruptr{40}{16}}
%% \put(0,0){\luptr{40}{16}}
%% \put(0,0){\lregptr{lreg}{10}}
%% \put(0,0){\rregptr{rreg}{10}}
%% \put(0,0){\luregptr{lreg}{10}}
%% \put(0,0){\ruregptr{rreg}{10}}
%% \put(0,0){\upetc}
%% \put(0,0){\downetc}
%% \put(0,0){\upbars}
%% \put(0,0){\downbars}
%% \put(0,0){\updownbars}
%%
%% \end{picture}
%

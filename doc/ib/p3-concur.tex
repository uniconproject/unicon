\chapter{Concurrency and Thread-Safety}

A {\em thread\/} is a fundamental build-ing block of an executing
program, consisting of a set of registers (including a program counter
indicating what instruction it is executing) and a stack with which
to evaluate intermediate results, place parameters and perform
function calls, etc.  If you have a running program, you have one
thread; if you have multiple threads, you might have one running at
a time and the threads taking turns, or you might have more than one
running simultaneously, a condition known as concurrent execution.

Icon's co-expression type is a thread-type that does not allow
concurrent execution.  In fact, the current implementation of Icon
uses posix threads with simple rules to ensure that only one of them
at a time can ever execute.

Since the turn of the 21st century, however, concurrency has become
essential to utilizing modern computer processors.  Once CPU
manufacturers hit physical limits and could no longer double the
clock speed every 18 months, they have turned a large part of their
attention to the addition of more and more execution cores at a given
clock speed.  Because multiple cores are now ubiquitous, and
concurrent programming is notoriously difficult, the pressure on
very-high level languages to support concurrency is high, but
adding concurrency to a very high-level language is more difficult
than it might seem.  Most popular mainstream scripting languages
offer pseudo-concurrency, but true concurrency is often limited by
a ``global interpreter lock'', or GIL, that prevents simultaneous
access by multiple threads to the same virtual machine.

In order to provide true concurrency, Unicon extends Icon's posix
threads co-expression implementation, and provides a thread-safe
runtime system.  Some of this was accomplished by switching out unsafe
C library functions for their thread-safe counterparts, but a far more
substantial part of it was accomplished by making threads as
independent of each other as possible, giving each thread its own copy
of various interpreter state variables that formerly were
globals. This is one of the few semantic differences between Icon and
Unicon.

\subsection{Thread State}

Unicon's threads implementation was influenced heavily by the
earlier development of dynamic loading and the ability to execute
multiple programs as co-expressions within the same virtual machine.
That earlier implementation produced a notion of a ``current program
state'' embodied by a \texttt{struct progstate}. Concurrent threads
share parts of what was the progstate, while giving each thread its
own copy of as many parts as possible in order to minimize threads'
interference with each other.  Those parts of the global state that
are replicated on a per-thred basis form \texttt{struct threadstate}
in src/h/rstructs.h.

\begin{iconcode}
\end{iconcode}
